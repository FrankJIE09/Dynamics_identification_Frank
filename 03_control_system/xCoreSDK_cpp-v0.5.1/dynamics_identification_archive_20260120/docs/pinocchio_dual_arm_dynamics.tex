\documentclass[12pt,a4paper]{article}
\usepackage[UTF8]{ctex}
\usepackage{amsmath}
\usepackage{amssymb}
\usepackage{geometry}
\usepackage{listings}
\usepackage{xcolor}
\usepackage{hyperref}
\usepackage{graphicx}
\usepackage{float}

\geometry{margin=2.5cm}

\title{Pinocchio 动力学计算详解:单臂、双臂与全身系统}
\author{基于 hybrid\_force\_position\_control.cpp}
\date{\today}

\lstset{
    language=C++,
    basicstyle=\ttfamily\small,
    keywordstyle=\color{blue}\bfseries,
    commentstyle=\color{green!60!black},
    stringstyle=\color{red},
    numbers=left,
    numberstyle=\tiny\color{gray},
    stepnumber=1,
    numbersep=5pt,
    frame=single,
    breaklines=true,
    showspaces=false,
    showstringspaces=false,
    tabsize=2
}

\begin{document}

\maketitle

\tableofcontents
\newpage

\section{概述}

本文档基于 \texttt{hybrid\_force\_position\_control.cpp} 中的实现,详细介绍 Pinocchio 库如何计算机器人动力学,并对比单臂系统和双臂系统的计算方法。

\section{Pinocchio 动力学计算基础}

\subsection{机器人动力学方程}

对于 $n$ 自由度的机器人,其动力学方程可以表示为:

\begin{equation}
\boldsymbol{\tau} = \mathbf{M}(\boldsymbol{q})\ddot{\boldsymbol{q}} + \mathbf{C}(\boldsymbol{q}, \dot{\boldsymbol{q}})\dot{\boldsymbol{q}} + \boldsymbol{g}(\boldsymbol{q}) + \boldsymbol{\tau}_f
\end{equation}

其中:
\begin{itemize}
    \item $\boldsymbol{\tau} \in \mathbb{R}^n$:关节力矩向量
    \item $\mathbf{M}(\boldsymbol{q}) \in \mathbb{R}^{n \times n}$:质量矩阵(惯性矩阵)
    \item $\mathbf{C}(\boldsymbol{q}, \dot{\boldsymbol{q}}) \in \mathbb{R}^{n \times n}$:科氏力矩阵
    \item $\boldsymbol{g}(\boldsymbol{q}) \in \mathbb{R}^n$:重力项
    \item $\boldsymbol{\tau}_f$:摩擦力项
    \item $\boldsymbol{q}, \dot{\boldsymbol{q}}, \ddot{\boldsymbol{q}} \in \mathbb{R}^n$:关节位置、速度、加速度
\end{itemize}

\subsection{Pinocchio 的核心函数}

Pinocchio 使用 \texttt{computeAllTerms()} 函数一次性计算所有动力学项:

\begin{lstlisting}[caption=Pinocchio 动力学计算]
// 计算所有动力学项
pinocchio::computeAllTerms(pinocchio_model, pinocchio_data, q_pin, dq_pin);

// 获取各项
gravity_pin = pinocchio_data.g;        // 重力项 g(q)
coriolis_pin = pinocchio_data.C;       // 科氏力矩阵 C(q, dq)
mass_pin = pinocchio_data.M;           // 质量矩阵 M(q)
coriolis_torque_pin = coriolis_pin * dq_pin;  // 科氏力扭矩 C(q, dq) * dq
inertia_torque_pin = mass_pin * ddq_pin;      // 惯性力扭矩 M(q) * ddq
\end{lstlisting}

\section{单臂系统动力学计算}

\subsection{模型初始化}

在 \texttt{hybrid\_force\_position\_control.cpp} 中,单臂系统(7自由度)的初始化过程如下:

\subsubsection{1. 加载 URDF 模型}

\begin{lstlisting}[caption=加载 URDF 模型]
// URDF 文件路径
std::string urdf_file_path = URDF_FILE_PATH;  // 编译时定义

// 创建模型和数据对象
pinocchio::Model pinocchio_model;
pinocchio::Data pinocchio_data(pinocchio_model);

// 从 URDF 构建模型
pinocchio::urdf::buildModel(urdf_file_path, pinocchio_model);
\end{lstlisting}

\textbf{关键点:}
\begin{itemize}
    \item URDF 文件定义了机器人的树形结构(从基座到末端执行器)
    \item \texttt{buildModel()} 解析 URDF,构建内部表示
    \item 对于单臂系统,\texttt{pinocchio\_model.nq = 7}(7个关节)
\end{itemize}

\subsubsection{2. 设置重力方向}

\begin{lstlisting}[caption=设置重力方向]
// 考虑安装位置:外旋 XYZ = 0 90 -90
Eigen::Matrix3d R_y = Eigen::AngleAxisd(M_PI/2, Eigen::Vector3d::UnitY()).toRotationMatrix();
Eigen::Matrix3d R_z = Eigen::AngleAxisd(-M_PI/2, Eigen::Vector3d::UnitZ()).toRotationMatrix();
Eigen::Matrix3d R_base_to_world = R_z * R_y;
Eigen::Vector3d gravity_world(0, 0, -9.81);
Eigen::Vector3d gravity_base = R_base_to_world.transpose() * gravity_world;

pinocchio_model.gravity.linear() = gravity_base;
\end{lstlisting}

\textbf{说明:}
\begin{itemize}
    \item 重力方向必须根据机器人的安装位置进行坐标变换
    \item 世界坐标系中的重力为 $(0, 0, -9.81)$ m/s²
    \item 需要转换到机器人基座坐标系
\end{itemize}

\subsubsection{3. 添加负载}

\begin{lstlisting}[caption=添加末端执行器负载]
// 创建负载惯性
Eigen::Vector3d load_com(load_centre[0], load_centre[1], load_centre[2]);
Eigen::Matrix3d inertia_matrix = Eigen::Matrix3d::Zero();
inertia_matrix(0, 0) = load_inertia[0];  // Ixx
inertia_matrix(1, 1) = load_inertia[1];  // Iyy
inertia_matrix(2, 2) = load_inertia[2];  // Izz
pinocchio::Inertia load_inertia_pin(load_mass, load_com, inertia_matrix);

// 将负载添加到末端执行器 joint 的惯性
pinocchio_model.inertias[ee_joint_id] += load_inertia_pin;
\end{lstlisting}

\subsection{实时控制循环中的计算}

在控制循环(1ms 周期)中,每次迭代都进行动力学计算:

\begin{lstlisting}[caption=实时动力学计算]
// 1. 获取当前关节状态(7维)
std::array<double, 7> q{}, dq_m{}, ddq_c{};
robot.getStateData(jointPos_m, q);      // 关节位置
robot.getStateData(jointVel_m, dq_m);   // 关节速度
robot.getStateData(jointAcc_c, ddq_c);  // 关节加速度

// 2. 转换为 Eigen 向量
Eigen::VectorXd q_pin(7), dq_pin(7), ddq_pin(7);
for (int i = 0; i < 7; i++) {
    q_pin(i) = q[i];
    dq_pin(i) = dq_m[i];
    ddq_pin(i) = ddq_c[i];
}

// 3. 计算所有动力学项
pinocchio::computeAllTerms(pinocchio_model, pinocchio_data, q_pin, dq_pin);

// 4. 提取各项
gravity_pin = pinocchio_data.g;                    // 7维重力项
coriolis_pin = pinocchio_data.C;                   // 7x7 科氏力矩阵
mass_pin = pinocchio_data.M;                       // 7x7 质量矩阵
coriolis_torque_pin = coriolis_pin * dq_pin;      // 7维科氏力扭矩
inertia_torque_pin = mass_pin * ddq_pin;           // 7维惯性力扭矩
\end{lstlisting}

\subsection{单臂系统的数学表示}

对于单臂系统($n=7$):

\begin{align}
\boldsymbol{q} &\in \mathbb{R}^7 \\
\mathbf{M}(\boldsymbol{q}) &\in \mathbb{R}^{7 \times 7} \\
\mathbf{C}(\boldsymbol{q}, \dot{\boldsymbol{q}}) &\in \mathbb{R}^{7 \times 7} \\
\boldsymbol{g}(\boldsymbol{q}) &\in \mathbb{R}^7
\end{align}

总动力学扭矩为:

\begin{equation}
\boldsymbol{\tau}_{\text{full}} = \mathbf{M}(\boldsymbol{q})\ddot{\boldsymbol{q}} + \mathbf{C}(\boldsymbol{q}, \dot{\boldsymbol{q}})\dot{\boldsymbol{q}} + \boldsymbol{g}(\boldsymbol{q})
\end{equation}

\section{双臂系统动力学计算}

\subsection{双臂系统的特点}

双臂系统与单臂系统的主要区别:

\begin{enumerate}
    \item \textbf{自由度增加:} 如果每条臂有 7 个关节,双臂系统共有 14 个关节
    \item \textbf{树形结构:} 从基座分支出两条独立的运动链
    \item \textbf{耦合效应:} 两个手臂通过基座耦合,质量矩阵包含交叉项
\end{enumerate}

\subsection{URDF 结构}

双臂系统的 URDF 通常具有以下结构:

\begin{lstlisting}[language=XML, caption=双臂 URDF 结构示例]
<robot name="dual_arm_robot">
    <link name="base_link"/>
    
    <!-- 左臂 -->
    <joint name="left_joint_1" type="revolute">
        <parent link="base_link"/>
        <child link="left_link_1"/>
    </joint>
    <!-- ... 左臂的其他关节 ... -->
    <link name="left_end_effector"/>
    
    <!-- 右臂 -->
    <joint name="right_joint_1" type="revolute">
        <parent link="base_link"/>
        <child link="right_link_1"/>
    </joint>
    <!-- ... 右臂的其他关节 ... -->
    <link name="right_end_effector"/>
</robot>
\end{lstlisting}

\subsection{模型初始化(双臂)}

双臂系统的初始化过程与单臂类似,但需要注意:

\begin{lstlisting}[caption=双臂系统初始化]
// 1. 加载双臂 URDF(与单臂相同)
pinocchio::urdf::buildModel(urdf_file_path, pinocchio_model);
// 此时 pinocchio_model.nq = 14(假设每条臂7个关节)

// 2. 设置重力方向(与单臂相同)
pinocchio_model.gravity.linear() = gravity_base;

// 3. 添加负载(需要分别为两个末端执行器添加)
// 左臂末端执行器
pinocchio::FrameIndex left_ee_id = pinocchio_model.getFrameId("left_end_effector");
pinocchio::JointIndex left_ee_joint = pinocchio_model.frames[left_ee_id].parent;
pinocchio_model.inertias[left_ee_joint] += left_load_inertia;

// 右臂末端执行器
pinocchio::FrameIndex right_ee_id = pinocchio_model.getFrameId("right_end_effector");
pinocchio::JointIndex right_ee_joint = pinocchio_model.frames[right_ee_id].parent;
pinocchio_model.inertias[right_ee_joint] += right_load_inertia;
\end{lstlisting}

\subsection{实时控制循环中的计算(双臂)}

双臂系统的动力学计算需要处理 14 维向量:

\begin{lstlisting}[caption=双臂系统实时动力学计算]
// 1. 获取关节状态(14维)
std::array<double, 14> q{}, dq_m{}, ddq_c{};
// 假设前7个是左臂,后7个是右臂
// q[0-6]: 左臂关节
// q[7-13]: 右臂关节

// 2. 转换为 Eigen 向量
Eigen::VectorXd q_pin(14), dq_pin(14), ddq_pin(14);
for (int i = 0; i < 14; i++) {
    q_pin(i) = q[i];
    dq_pin(i) = dq_m[i];
    ddq_pin(i) = ddq_c[i];
}

// 3. 计算所有动力学项(与单臂相同)
pinocchio::computeAllTerms(pinocchio_model, pinocchio_data, q_pin, dq_pin);

// 4. 提取各项(维度变为14)
gravity_pin = pinocchio_data.g;                    // 14维重力项
coriolis_pin = pinocchio_data.C;                   // 14x14 科氏力矩阵
mass_pin = pinocchio_data.M;                       // 14x14 质量矩阵
coriolis_torque_pin = coriolis_pin * dq_pin;      // 14维科氏力扭矩
inertia_torque_pin = mass_pin * ddq_pin;           // 14维惯性力扭矩
\end{lstlisting}

\subsection{双臂系统的数学表示}

对于双臂系统(每条臂 $n=7$,总共 $N=14$):

\begin{align}
\boldsymbol{q} &= \begin{bmatrix} \boldsymbol{q}_L \\ \boldsymbol{q}_R \end{bmatrix} \in \mathbb{R}^{14} \\
\mathbf{M}(\boldsymbol{q}) &= \begin{bmatrix} 
    \mathbf{M}_{LL} & \mathbf{M}_{LR} \\
    \mathbf{M}_{RL} & \mathbf{M}_{RR}
\end{bmatrix} \in \mathbb{R}^{14 \times 14} \\
\mathbf{C}(\boldsymbol{q}, \dot{\boldsymbol{q}}) &\in \mathbb{R}^{14 \times 14} \\
\boldsymbol{g}(\boldsymbol{q}) &= \begin{bmatrix} \boldsymbol{g}_L \\ \boldsymbol{g}_R \end{bmatrix} \in \mathbb{R}^{14}
\end{align}

其中:
\begin{itemize}
    \item $\boldsymbol{q}_L, \boldsymbol{q}_R \in \mathbb{R}^7$:左臂和右臂的关节角度
    \item $\mathbf{M}_{LL}, \mathbf{M}_{RR} \in \mathbb{R}^{7 \times 7}$:左臂和右臂的自惯性矩阵
    \item $\mathbf{M}_{LR}, \mathbf{M}_{RL} \in \mathbb{R}^{7 \times 7}$:双臂之间的耦合惯性矩阵
\end{itemize}

\subsection{双臂系统的耦合效应}

双臂系统的质量矩阵包含耦合项,反映了两个手臂之间的动力学耦合:

\begin{equation}
\mathbf{M}(\boldsymbol{q}) = \begin{bmatrix} 
    \mathbf{M}_{LL}(\boldsymbol{q}) & \mathbf{M}_{LR}(\boldsymbol{q}) \\
    \mathbf{M}_{RL}(\boldsymbol{q}) & \mathbf{M}_{RR}(\boldsymbol{q})
\end{bmatrix}
\end{equation}

\textbf{物理意义:}
\begin{itemize}
    \item $\mathbf{M}_{LL}$:左臂运动时,左臂自身的惯性
    \item $\mathbf{M}_{RR}$:右臂运动时,右臂自身的惯性
    \item $\mathbf{M}_{LR}$:右臂运动对左臂的惯性影响(通过基座耦合)
    \item $\mathbf{M}_{RL}$:左臂运动对右臂的惯性影响(通过基座耦合)
\end{itemize}

\section{代码实现对比}

\subsection{单臂系统代码结构}

\begin{lstlisting}[caption=单臂系统关键代码]
// 模型维度
const int n_dof = 7;

// 状态向量
std::array<double, n_dof> q{}, dq_m{}, ddq_c{};

// Pinocchio 计算
pinocchio::computeAllTerms(pinocchio_model, pinocchio_data, q_pin, dq_pin);

// 结果维度
// gravity_pin: 7维
// mass_pin: 7x7
// coriolis_pin: 7x7
\end{lstlisting}

\subsection{双臂系统代码结构}

\begin{lstlisting}[caption=双臂系统关键代码]
// 模型维度
const int n_dof_per_arm = 7;
const int n_dof_total = 14;

// 状态向量(需要拼接两个手臂的状态)
std::array<double, n_dof_total> q{}, dq_m{}, ddq_c{};
// q[0-6]: 左臂关节
// q[7-13]: 右臂关节

// Pinocchio 计算(与单臂完全相同)
pinocchio::computeAllTerms(pinocchio_model, pinocchio_data, q_pin, dq_pin);

// 结果维度(自动变为14维)
// gravity_pin: 14维
// mass_pin: 14x14
// coriolis_pin: 14x14
\end{lstlisting}

\section{雅可比矩阵计算}

\subsection{单臂系统雅可比}

单臂系统只需要计算一个末端执行器的雅可比矩阵:

\begin{lstlisting}[caption=单臂雅可比计算]
// 获取末端执行器索引
pinocchio::FrameIndex ee_id = pinocchio_model.getFrameId("end_effector");

// 计算雅可比矩阵(6x7)
Eigen::MatrixXd J = pinocchio::computeFrameJacobian(
    pinocchio_model, pinocchio_data, q_pin, ee_id, 
    pinocchio::LOCAL_WORLD_ALIGNED);
// J: 6x7 矩阵(6维任务空间,7维关节空间)
\end{lstlisting}

\subsection{双臂系统雅可比}

双臂系统需要分别计算两个末端执行器的雅可比矩阵:

\begin{lstlisting}[caption=双臂雅可比计算]
// 获取两个末端执行器索引
pinocchio::FrameIndex left_ee_id = pinocchio_model.getFrameId("left_end_effector");
pinocchio::FrameIndex right_ee_id = pinocchio_model.getFrameId("right_end_effector");

// 计算左臂雅可比(6x14,但只有前7列非零)
Eigen::MatrixXd J_left = pinocchio::computeFrameJacobian(
    pinocchio_model, pinocchio_data, q_pin, left_ee_id, 
    pinocchio::LOCAL_WORLD_ALIGNED);
// J_left: 6x14 矩阵

// 计算右臂雅可比(6x14,但只有后7列非零)
Eigen::MatrixXd J_right = pinocchio::computeFrameJacobian(
    pinocchio_model, pinocchio_data, q_pin, right_ee_id, 
    pinocchio::LOCAL_WORLD_ALIGNED);
// J_right: 6x14 矩阵

// 组合雅可比(用于双臂协同控制)
Eigen::MatrixXd J_combined(12, 14);
J_combined << J_left, J_right;  // 12x14 矩阵
\end{lstlisting}

\section{任务空间质量矩阵}

\subsection{单臂系统任务空间质量矩阵}

对于单臂系统,任务空间质量矩阵为:

\begin{equation}
\boldsymbol{\Lambda}_s = (\mathbf{J}_s \mathbf{M}^{-1} \mathbf{J}_s^T)^{-1} \in \mathbb{R}^{6 \times 6}
\end{equation}

其中 $\mathbf{J}_s \in \mathbb{R}^{6 \times 7}$ 是任务空间雅可比矩阵。

\subsection{双臂系统任务空间质量矩阵}

对于双臂系统,有两种处理方式:

\subsubsection{方式1:分别计算每个手臂的任务空间质量矩阵}

\begin{align}
\boldsymbol{\Lambda}_{L} &= (\mathbf{J}_L \mathbf{M}^{-1} \mathbf{J}_L^T)^{-1} \in \mathbb{R}^{6 \times 6} \\
\boldsymbol{\Lambda}_{R} &= (\mathbf{J}_R \mathbf{M}^{-1} \mathbf{J}_R^T)^{-1} \in \mathbb{R}^{6 \times 6}
\end{align}

\subsubsection{方式2:计算组合任务空间质量矩阵}

如果定义组合任务空间(12维:左臂6维 + 右臂6维):

\begin{equation}
\boldsymbol{\Lambda}_{\text{combined}} = (\mathbf{J}_{\text{combined}} \mathbf{M}^{-1} \mathbf{J}_{\text{combined}}^T)^{-1} \in \mathbb{R}^{12 \times 12}
\end{equation}

其中 $\mathbf{J}_{\text{combined}} = \begin{bmatrix} \mathbf{J}_L \\ \mathbf{J}_R \end{bmatrix} \in \mathbb{R}^{12 \times 14}$。

\section{计算复杂度分析}

\subsection{单臂系统}

\begin{itemize}
    \item \textbf{质量矩阵计算:} $O(n^2)$,其中 $n=7$,约为 $O(49)$
    \item \textbf{科氏力矩阵计算:} $O(n^2)$
    \item \textbf{重力项计算:} $O(n)$
    \item \textbf{总复杂度:} $O(n^2)$
\end{itemize}

\subsection{双臂系统}

\begin{itemize}
    \item \textbf{质量矩阵计算:} $O(N^2)$,其中 $N=14$,约为 $O(196)$
    \item \textbf{科氏力矩阵计算:} $O(N^2)$
    \item \textbf{重力项计算:} $O(N)$
    \item \textbf{总复杂度:} $O(N^2)$
\end{itemize}

\textbf{注意:} 双臂系统的计算复杂度是单臂的 4 倍($14^2 / 7^2 = 4$),但 Pinocchio 使用高效的递归算法(RNEA),实际计算时间可能小于 4 倍。

\section{实际应用建议}

\subsection{单臂系统}

\begin{enumerate}
    \item 直接使用 \texttt{computeAllTerms()} 计算所有动力学项
    \item 计算单个末端执行器的雅可比矩阵
    \item 使用标准的混合力位控制算法
\end{enumerate}

\subsection{双臂系统}

\begin{enumerate}
    \item \textbf{状态向量组织:} 确保关节状态按照 URDF 中定义的顺序排列
    \item \textbf{分别处理:} 可以分别计算每个手臂的雅可比和任务空间质量矩阵
    \item \textbf{耦合考虑:} 如果需要精确控制,需要考虑质量矩阵中的耦合项
    \item \textbf{协同控制:} 可以使用组合雅可比矩阵实现双臂协同控制
\end{enumerate}

\section{全身系统动力学计算}

\subsection{全身系统的结构}

全身系统通常包括以下部分:

\begin{enumerate}
    \item \textbf{底盘(Base):} 移动平台,可能包含轮式或足式移动机构
    \item \textbf{腰部(Torso):} 连接底盘和上身的中间部分
        \begin{itemize}
            \item \textbf{折叠关节(Pitch):} 前后俯仰运动
            \item \textbf{旋转关节(Yaw):} 左右旋转运动
            \item 可能还有侧倾关节(Roll)
        \end{itemize}
    \item \textbf{双臂:} 左臂和右臂,每条臂通常有 7 个关节
    \item \textbf{头部(可选):} 可能包含视觉传感器
\end{enumerate}

\subsection{URDF 结构示例}

全身系统的 URDF 结构如下:

\begin{lstlisting}[language=XML, caption=全身系统 URDF 结构]
<robot name="full_body_robot">
    <!-- 世界坐标系 -->
    <link name="world"/>
    
    <!-- 底盘 -->
    <link name="base_link">
        <inertial>
            <mass value="50.0"/>  <!-- 底盘质量 -->
            <inertia ixx="10.0" iyy="10.0" izz="5.0"/>
        </inertial>
    </link>
    
    <!-- 底盘到世界的连接(固定或浮动) -->
    <joint name="base_joint" type="floating">
        <parent link="world"/>
        <child link="base_link"/>
    </joint>
    
    <!-- 腰部折叠关节(Pitch) -->
    <joint name="torso_pitch_joint" type="revolute">
        <parent link="base_link"/>
        <child link="torso_link"/>
        <axis xyz="0 1 0"/>  <!-- Y轴旋转,前后俯仰 -->
        <limit lower="-1.57" upper="1.57"/>  <!-- ±90度 -->
    </joint>
    
    <link name="torso_link">
        <inertial>
            <mass value="20.0"/>  <!-- 腰部质量 -->
            <inertia ixx="2.0" iyy="2.0" izz="1.0"/>
        </inertial>
    </link>
    
    <!-- 腰部旋转关节(Yaw) -->
    <joint name="torso_yaw_joint" type="revolute">
        <parent link="torso_link"/>
        <child link="torso_upper_link"/>
        <axis xyz="0 0 1"/>  <!-- Z轴旋转,左右旋转 -->
        <limit lower="-3.14" upper="3.14"/>  <!-- ±180度 -->
    </joint>
    
    <link name="torso_upper_link">
        <inertial>
            <mass value="15.0"/>
            <inertia ixx="1.5" iyy="1.5" izz="0.8"/>
        </inertial>
    </link>
    
    <!-- 左臂(从腰部上端连接) -->
    <joint name="left_shoulder_joint" type="revolute">
        <parent link="torso_upper_link"/>
        <child link="left_arm_link1"/>
    </joint>
    <!-- ... 左臂的其他关节 ... -->
    
    <!-- 右臂(从腰部上端连接) -->
    <joint name="right_shoulder_joint" type="revolute">
        <parent link="torso_upper_link"/>
        <child link="right_arm_link1"/>
    </joint>
    <!-- ... 右臂的其他关节 ... -->
</robot>
\end{lstlisting}

\subsection{全身系统的自由度}

假设系统配置如下:
\begin{itemize}
    \item 底盘:6 自由度(3 个位置 + 3 个姿态,如果是浮动基座)
    \item 腰部折叠关节:1 自由度(Pitch)
    \item 腰部旋转关节:1 自由度(Yaw)
    \item 左臂:7 自由度
    \item 右臂:7 自由度
    \item \textbf{总自由度:} $N = 6 + 1 + 1 + 7 + 7 = 22$ 自由度
\end{itemize}

\textbf{注意:} 如果底盘是固定的(非浮动),则底盘自由度为 0,总自由度为 $N = 1 + 1 + 7 + 7 = 16$。

\subsection{模型初始化(全身系统)}

\begin{lstlisting}[caption=全身系统初始化]
// 1. 加载全身 URDF
pinocchio::urdf::buildModel(urdf_file_path, pinocchio_model);
// 此时 pinocchio_model.nq = 22(假设浮动基座)

// 2. 设置重力方向
pinocchio_model.gravity.linear() = Eigen::Vector3d(0, 0, -9.81);

// 3. 获取关键关节索引
// 腰部折叠关节
int torso_pitch_id = pinocchio_model.getJointId("torso_pitch_joint");
// 腰部旋转关节
int torso_yaw_id = pinocchio_model.getJointId("torso_yaw_joint");
// 左臂肩关节
int left_shoulder_id = pinocchio_model.getJointId("left_shoulder_joint");
// 右臂肩关节
int right_shoulder_id = pinocchio_model.getJointId("right_shoulder_joint");

// 4. 添加负载(分别为两个末端执行器)
pinocchio::FrameIndex left_ee_id = pinocchio_model.getFrameId("left_end_effector");
pinocchio::FrameIndex right_ee_id = pinocchio_model.getFrameId("right_end_effector");
// ... 添加负载惯性 ...
\end{lstlisting}

\subsection{实时控制循环中的计算(全身系统)}

全身系统的状态向量组织方式:

\begin{lstlisting}[caption=全身系统状态向量组织]
// 状态向量维度:22(假设浮动基座)
std::array<double, 22> q{}, dq_m{}, ddq_c{};

// 向量组织方式(按 URDF 中定义的顺序):
// q[0-5]:   底盘位置和姿态(如果是浮动基座)
//           [x, y, z, roll, pitch, yaw]
// q[6]:     腰部折叠关节(Pitch)
// q[7]:     腰部旋转关节(Yaw)
// q[8-14]:  左臂关节(7个)
// q[15-21]: 右臂关节(7个)

// 转换为 Eigen 向量
Eigen::VectorXd q_pin(22), dq_pin(22), ddq_pin(22);
for (int i = 0; i < 22; i++) {
    q_pin(i) = q[i];
    dq_pin(i) = dq_m[i];
    ddq_pin(i) = ddq_c[i];
}

// 计算所有动力学项(与单臂/双臂完全相同)
pinocchio::computeAllTerms(pinocchio_model, pinocchio_data, q_pin, dq_pin);

// 提取各项(维度变为22)
gravity_pin = pinocchio_data.g;                    // 22维重力项
coriolis_pin = pinocchio_data.C;                   // 22x22 科氏力矩阵
mass_pin = pinocchio_data.M;                       // 22x22 质量矩阵
coriolis_torque_pin = coriolis_pin * dq_pin;      // 22维科氏力扭矩
inertia_torque_pin = mass_pin * ddq_pin;           // 22维惯性力扭矩
\end{lstlisting}

\subsection{全身系统的数学表示}

对于全身系统,质量矩阵具有分块结构:

\begin{equation}
\mathbf{M}(\boldsymbol{q}) = \begin{bmatrix}
    \mathbf{M}_{BB} & \mathbf{M}_{BT} & \mathbf{M}_{BL} & \mathbf{M}_{BR} \\
    \mathbf{M}_{TB} & \mathbf{M}_{TT} & \mathbf{M}_{TL} & \mathbf{M}_{TR} \\
    \mathbf{M}_{LB} & \mathbf{M}_{LT} & \mathbf{M}_{LL} & \mathbf{M}_{LR} \\
    \mathbf{M}_{RB} & \mathbf{M}_{RT} & \mathbf{M}_{RL} & \mathbf{M}_{RR}
\end{bmatrix} \in \mathbb{R}^{22 \times 22}
\end{equation}

其中:
\begin{itemize}
    \item $\mathbf{M}_{BB} \in \mathbb{R}^{6 \times 6}$:底盘的惯性矩阵
    \item $\mathbf{M}_{TT} \in \mathbb{R}^{2 \times 2}$:腰部的惯性矩阵(折叠+旋转)
    \item $\mathbf{M}_{LL}, \mathbf{M}_{RR} \in \mathbb{R}^{7 \times 7}$:左臂和右臂的自惯性矩阵
    \item 其他块表示不同部分之间的耦合惯性
\end{itemize}

\subsection{腰部关节的特殊性}

\subsubsection{折叠关节(Pitch)的影响}

腰部折叠关节的运动会影响:
\begin{enumerate}
    \item \textbf{重力补偿:} 改变整个上身(包括双臂)的重心位置
    \item \textbf{惯性耦合:} 当腰部折叠时,双臂的惯性会发生变化
    \item \textbf{科氏力:} 腰部折叠速度会影响双臂的科氏力项
\end{enumerate}

数学表示:

\begin{equation}
\boldsymbol{g}(\boldsymbol{q}) = \begin{bmatrix}
    \boldsymbol{g}_B \\
    \boldsymbol{g}_T(\boldsymbol{q}_T, \boldsymbol{q}_L, \boldsymbol{q}_R) \\
    \boldsymbol{g}_L(\boldsymbol{q}_T, \boldsymbol{q}_L) \\
    \boldsymbol{g}_R(\boldsymbol{q}_T, \boldsymbol{q}_R)
\end{bmatrix}
\end{equation}

其中 $\boldsymbol{g}_T, \boldsymbol{g}_L, \boldsymbol{g}_R$ 都依赖于腰部折叠角度 $\boldsymbol{q}_T$。

\subsubsection{旋转关节(Yaw)的影响}

腰部旋转关节主要影响:
\begin{enumerate}
    \item \textbf{坐标系变换:} 改变双臂相对于世界坐标系的方向
    \item \textbf{科氏力耦合:} 旋转速度会产生额外的科氏力项
    \item \textbf{任务空间映射:} 影响末端执行器在世界坐标系中的位置
\end{enumerate}

\subsection{雅可比矩阵计算(全身系统)}

对于全身系统,需要分别计算各个末端执行器的雅可比矩阵:

\begin{lstlisting}[caption=全身系统雅可比计算]
// 计算左臂末端执行器雅可比(6x22)
pinocchio::FrameIndex left_ee_id = pinocchio_model.getFrameId("left_end_effector");
Eigen::MatrixXd J_left = pinocchio::computeFrameJacobian(
    pinocchio_model, pinocchio_data, q_pin, left_ee_id, 
    pinocchio::LOCAL_WORLD_ALIGNED);
// J_left 的列对应:底盘(6) + 腰部(2) + 左臂(7) + 右臂(7)
// 但只有前 6+2+7=15 列非零

// 计算右臂末端执行器雅可比(6x22)
pinocchio::FrameIndex right_ee_id = pinocchio_model.getFrameId("right_end_effector");
Eigen::MatrixXd J_right = pinocchio::computeFrameJacobian(
    pinocchio_model, pinocchio_data, q_pin, right_ee_id, 
    pinocchio::LOCAL_WORLD_ALIGNED);
// J_right 的列对应:底盘(6) + 腰部(2) + 左臂(7) + 右臂(7)
// 但只有前 6+2+7=15 列和后 7 列非零

// 计算腰部中心点雅可比(可选,用于腰部控制)
pinocchio::FrameIndex torso_id = pinocchio_model.getFrameId("torso_upper_link");
Eigen::MatrixXd J_torso = pinocchio::computeFrameJacobian(
    pinocchio_model, pinocchio_data, q_pin, torso_id, 
    pinocchio::LOCAL_WORLD_ALIGNED);
\end{lstlisting}

\subsection{任务空间质量矩阵(全身系统)}

对于全身系统,任务空间质量矩阵的计算需要考虑所有耦合:

\begin{equation}
\boldsymbol{\Lambda}_L = (\mathbf{J}_L \mathbf{M}^{-1} \mathbf{J}_L^T)^{-1} \in \mathbb{R}^{6 \times 6}
\end{equation}

其中 $\mathbf{J}_L \in \mathbb{R}^{6 \times 22}$ 是左臂的雅可比矩阵。

\textbf{关键点:} 由于 $\mathbf{J}_L$ 包含了底盘、腰部和左臂的列,任务空间质量矩阵会自动考虑这些部分的耦合效应。

\subsection{控制策略考虑}

\subsubsection{分层控制}

对于全身系统,可以采用分层控制策略:

\begin{enumerate}
    \item \textbf{底层:} 控制底盘和腰部,维持整体平衡
    \item \textbf{中层:} 控制腰部姿态,优化双臂工作空间
    \item \textbf{顶层:} 控制双臂末端执行器,执行具体任务
\end{enumerate}

\subsubsection{优先级控制}

可以使用零空间投影实现优先级控制:

\begin{lstlisting}[caption=优先级控制示例]
// 主要任务:控制左臂末端位置
Eigen::VectorXd F_primary = computeLeftArmForce();

// 次要任务:控制腰部姿态(在左臂任务的零空间中)
Eigen::MatrixXd J_left = computeLeftArmJacobian();
Eigen::MatrixXd P_null = Eigen::MatrixXd::Identity(22, 22) - 
                         J_left.transpose() * (J_left * J_left.transpose()).inverse() * J_left;
Eigen::VectorXd F_secondary = computeTorsoPostureForce();
Eigen::VectorXd tau_secondary = P_null * J_torso.transpose() * F_secondary;

// 总控制力矩
Eigen::VectorXd tau = J_left.transpose() * F_primary + tau_secondary;
\end{lstlisting}

\subsection{计算复杂度(全身系统)}

对于全身系统($N=22$ 自由度):

\begin{itemize}
    \item \textbf{质量矩阵计算:} $O(N^2) = O(484)$
    \item \textbf{科氏力矩阵计算:} $O(N^2)$
    \item \textbf{重力项计算:} $O(N)$
    \item \textbf{总复杂度:} $O(N^2)$
\end{itemize}

\textbf{对比:}
\begin{itemize}
    \item 单臂(7自由度):$O(49)$
    \item 双臂(14自由度):$O(196)$
    \item 全身(22自由度):$O(484)$
\end{itemize}

虽然计算复杂度增加,但 Pinocchio 的高效递归算法(RNEA)仍能满足实时控制需求(通常 $< 1$ ms)。

\section{总结}

\begin{itemize}
    \item \textbf{Pinocchio 完全支持全身系统:} 包括底盘、腰部(折叠+旋转)、双臂等复杂结构
    \item \textbf{统一的计算接口:} 无论系统多复杂,都使用相同的 \texttt{computeAllTerms()} 函数
    \item \textbf{自动处理耦合:} Pinocchio 自动计算所有部分之间的动力学耦合
    \item \textbf{代码改动最小:} 主要改动是向量维度,核心计算逻辑不变
    \item \textbf{腰部关节的影响:} 折叠和旋转关节会影响整个上身的动力学,这些影响会自动包含在计算结果中
    \item \textbf{性能考虑:} 全身系统的计算复杂度较高,但 Pinocchio 的高效实现仍能满足实时控制需求
\end{itemize}

\section{参考文献}

\begin{enumerate}
    \item Pinocchio Documentation: \url{https://stack-of-tasks.github.io/pinocchio/}
    \item Featherstone, R. (2008). \textit{Rigid Body Dynamics Algorithms}. Springer.
    \item \texttt{hybrid\_force\_position\_control.cpp} - ROKAE xCore SDK 示例代码
\end{enumerate}

\end{document}

