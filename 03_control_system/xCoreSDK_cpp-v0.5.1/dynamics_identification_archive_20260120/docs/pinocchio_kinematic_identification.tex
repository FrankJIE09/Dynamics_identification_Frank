\documentclass[12pt,a4paper]{article}
\usepackage[UTF8]{ctex}
\usepackage{amsmath}
\usepackage{amssymb}
\usepackage{geometry}
\usepackage{listings}
\usepackage{xcolor}
\usepackage{hyperref}
\usepackage{graphicx}
\usepackage{float}
\usepackage{algorithm}
\usepackage{algorithmic}

\geometry{margin=2.5cm}

\title{Pinocchio 运动学参数辨识详解}
\author{基于 hybrid\_force\_position\_control.cpp 和 xCore SDK}
\date{\today}

\lstset{
    language=C++,
    basicstyle=\ttfamily\small,
    keywordstyle=\color{blue}\bfseries,
    commentstyle=\color{green!60!black},
    stringstyle=\color{red},
    numbers=left,
    numberstyle=\tiny\color{gray},
    stepnumber=1,
    numbersep=5pt,
    frame=single,
    breaklines=true,
    showspaces=false,
    showstringspaces=false,
    tabsize=2
}

\begin{document}

\maketitle

\tableofcontents
\newpage

\section{概述}

运动学参数辨识是机器人标定的重要环节,通过实验数据估计机器人的几何参数(连杆长度、关节偏置、关节轴方向等),以提高定位精度。本文档介绍如何使用 Pinocchio 库进行运动学参数辨识。

\section{运动学辨识的基本原理}

\subsection{待辨识的运动学参数}

对于机器人运动学,需要辨识的参数包括:

\begin{enumerate}
    \item \textbf{连杆长度:} $a_i$(相邻关节轴之间的最短距离)
    \item \textbf{连杆扭转角:} $\alpha_i$(相邻关节轴之间的扭转角)
    \item \textbf{关节偏置:} $d_i$(沿关节轴方向的偏移)
    \item \textbf{关节角度偏置:} $\theta_{i,0}$(关节零位偏置)
    \item \textbf{末端执行器位姿:} TCP(Tool Center Point)相对于最后一个关节的位姿
    \item \textbf{基坐标系位姿:} 机器人基座相对于世界坐标系的位置和姿态
\end{enumerate}

\subsection{正向运动学模型}

机器人的正向运动学可以表示为:

\begin{equation}
\mathbf{T}_{ee} = f(\boldsymbol{q}, \boldsymbol{\theta}_k)
\label{eq:forward_kinematics}
\end{equation}

其中:
\begin{itemize}
    \item $\mathbf{T}_{ee} \in SE(3)$:末端执行器相对于基坐标系的齐次变换矩阵
    \item $\boldsymbol{q} \in \mathbb{R}^n$:关节角度向量
    \item $\boldsymbol{\theta}_k \in \mathbb{R}^p$:待辨识的运动学参数向量
\end{itemize}

\subsection{线性化与回归矩阵}

对于小的参数误差,可以将运动学方程线性化:

\begin{equation}
\delta \mathbf{x}_{ee} = \mathbf{J}_k(\boldsymbol{q}, \boldsymbol{\theta}_k) \delta \boldsymbol{\theta}_k
\label{eq:kinematic_regressor}
\end{equation}

其中:
\begin{itemize}
    \item $\delta \mathbf{x}_{ee} \in \mathbb{R}^6$:末端执行器位姿误差(3维位置 + 3维姿态)
    \item $\mathbf{J}_k \in \mathbb{R}^{6 \times p}$:运动学回归矩阵(Kinematic Regressor)
    \item $\delta \boldsymbol{\theta}_k$:参数误差
\end{itemize}

\section{Pinocchio 的运动学辨识功能}

\subsection{核心函数}

Pinocchio 提供了以下关键函数用于运动学辨识:

\begin{lstlisting}[caption=Pinocchio 运动学辨识相关函数]
// 1. 计算正向运动学
pinocchio::forwardKinematics(model, data, q);
pinocchio::updateFramePlacements(model, data);

// 2. 计算末端执行器位姿
Eigen::Matrix4d T_ee = data.oMf[ee_frame_id].toHomogeneousMatrix();

// 3. 计算运动学回归矩阵(对参数求偏导)
pinocchio::computeFrameKinematicRegressor(
    model, data, frame_id, regressor);

// 4. 计算雅可比矩阵(对关节角度求偏导)
pinocchio::computeFrameJacobian(
    model, data, q, frame_id, pinocchio::LOCAL_WORLD_ALIGNED, J);

// 5. 计算关节配置的偏导数
pinocchio::computeJointKinematicRegressor(
    model, data, joint_id, regressor);
\end{lstlisting}

\subsection{运动学回归矩阵}

运动学回归矩阵 $\mathbf{J}_k$ 表示末端执行器位姿对运动学参数的敏感度:

\begin{equation}
\mathbf{J}_k = \frac{\partial \mathbf{x}_{ee}}{\partial \boldsymbol{\theta}_k}
\end{equation}

每一列对应一个参数的偏导数。

\section{辨识流程}

\subsection{步骤1:数据采集}

需要采集的数据包括:

\begin{lstlisting}[caption=运动学辨识数据采集]
struct KinematicDataPoint {
    std::vector<double> q;           // 关节角度 (n维)
    Eigen::Vector3d pos_measured;    // 末端位置(外部测量,如视觉、激光追踪)
    Eigen::Quaterniond quat_measured; // 末端姿态(外部测量)
    double timestamp;                // 时间戳
};

// 采集多个数据点
std::vector<KinematicDataPoint> collected_data;
\end{lstlisting}

\textbf{数据采集要求:}
\begin{itemize}
    \item 使用外部测量设备(视觉系统、激光追踪器、测量臂等)
    \item 覆盖机器人的整个工作空间
    \item 包含不同的关节角度组合
    \item 数据量要足够(通常需要数百到数千个数据点)
    \item 测量精度要足够高(通常要求位置误差 < 0.1mm,姿态误差 < 0.1°)
\end{itemize}

\subsection{步骤2:构造误差方程}

对于每个数据点,计算预测位姿和测量位姿的误差:

\begin{lstlisting}[caption=构造误差方程]
// 初始化
Eigen::MatrixXd J_all;      // 所有回归矩阵堆叠
Eigen::VectorXd error_all;   // 所有位姿误差堆叠

for (const auto& point : collected_data) {
    // 1. 使用当前参数计算正向运动学
    Eigen::VectorXd q = Eigen::Map<const Eigen::VectorXd>(
        point.q.data(), point.q.size());
    
    pinocchio::forwardKinematics(model, data, q);
    pinocchio::updateFramePlacements(model, data);
    
    // 2. 获取预测的末端执行器位姿
    Eigen::Matrix4d T_predicted = 
        data.oMf[ee_frame_id].toHomogeneousMatrix();
    Eigen::Vector3d pos_predicted = T_predicted.block<3,1>(0,3);
    Eigen::Matrix3d R_predicted = T_predicted.block<3,3>(0,0);
    Eigen::Quaterniond quat_predicted(R_predicted);
    
    // 3. 计算位姿误差
    Eigen::Vector3d pos_error = point.pos_measured - pos_predicted;
    
    // 姿态误差(使用旋转向量)
    Eigen::Quaterniond quat_error = 
        point.quat_measured * quat_predicted.inverse();
    if (quat_error.w() < 0) {
        quat_error.coeffs() = -quat_error.coeffs();
    }
    Eigen::Vector3d rot_error = quat_error.vec();
    
    // 组合成6维误差向量
    Eigen::VectorXd error(6);
    error.head(3) = rot_error;      // 姿态误差
    error.tail(3) = pos_error;      // 位置误差
    
    // 4. 计算运动学回归矩阵
    Eigen::MatrixXd J_k(6, n_params);
    pinocchio::computeFrameKinematicRegressor(
        model, data, ee_frame_id, J_k);
    
    // 5. 堆叠
    if (J_all.rows() == 0) {
        J_all = J_k;
        error_all = error;
    } else {
        Eigen::MatrixXd J_new(J_all.rows() + 6, n_params);
        J_new << J_all, J_k;
        J_all = J_new;
        
        Eigen::VectorXd error_new(error_all.size() + 6);
        error_new << error_all, error;
        error_all = error_new;
    }
}
\end{lstlisting}

\subsection{步骤3:求解参数}

使用最小二乘法求解参数误差:

\begin{equation}
\delta \hat{\boldsymbol{\theta}}_k = (\mathbf{J}_k^T \mathbf{J}_k)^{-1} \mathbf{J}_k^T \boldsymbol{\epsilon}
\label{eq:kinematic_least_squares}
\end{equation}

其中 $\boldsymbol{\epsilon}$ 是所有位姿误差的堆叠向量。

\begin{lstlisting}[caption=求解运动学参数]
// 求解参数误差
Eigen::VectorXd delta_theta = 
    (J_all.transpose() * J_all).ldlt().solve(
        J_all.transpose() * error_all);

// 更新参数
Eigen::VectorXd theta_new = theta_initial + delta_theta;

// 更新模型
updateModelKinematicParameters(model, theta_new);
\end{lstlisting}

\subsection{步骤4:迭代优化}

由于运动学方程是非线性的,通常需要迭代优化:

\begin{lstlisting}[caption=迭代优化]
Eigen::VectorXd theta = theta_initial;
double tolerance = 1e-6;
int max_iterations = 100;

for (int iter = 0; iter < max_iterations; iter++) {
    // 1. 计算当前参数下的误差
    Eigen::MatrixXd J_all;
    Eigen::VectorXd error_all;
    constructErrorEquations(model, data, collected_data, 
                           J_all, error_all);
    
    // 2. 计算参数更新
    Eigen::VectorXd delta_theta = 
        (J_all.transpose() * J_all + lambda * I).ldlt().solve(
            J_all.transpose() * error_all);
    
    // 3. 更新参数
    theta += delta_theta;
    updateModelKinematicParameters(model, theta);
    
    // 4. 检查收敛
    if (delta_theta.norm() < tolerance) {
        std::cout << "收敛于迭代 " << iter << std::endl;
        break;
    }
}
\end{lstlisting}

\section{单臂系统运动学辨识}

\subsection{DH 参数辨识}

对于使用 DH(Denavit-Hartenberg)参数描述的机器人,需要辨识:

\begin{equation}
\boldsymbol{\theta}_k = [a_1, \alpha_1, d_1, \theta_{1,0}, \ldots, a_n, \alpha_n, d_n, \theta_{n,0}]^T
\end{equation}

\subsection{代码示例}

\begin{lstlisting}[caption=单臂系统运动学辨识]
void identifySingleArmKinematics(
    pinocchio::Model& model,
    const std::vector<KinematicDataPoint>& data) {
    
    pinocchio::Data data_pin(model);
    int ee_frame_id = model.getFrameId("end_effector");
    int n_params = getKinematicParameterCount(model);
    
    // 初始化参数
    Eigen::VectorXd theta = getInitialKinematicParameters(model);
    
    // 迭代优化
    for (int iter = 0; iter < 100; iter++) {
        Eigen::MatrixXd J_all;
        Eigen::VectorXd error_all;
        
        // 对每个数据点
        for (const auto& point : data) {
            Eigen::VectorXd q(point.q.size());
            for (size_t i = 0; i < point.q.size(); i++) {
                q(i) = point.q[i];
            }
            
            // 更新模型参数
            updateModelWithParameters(model, theta);
            
            // 计算正向运动学
            pinocchio::forwardKinematics(model, data_pin, q);
            pinocchio::updateFramePlacements(model, data_pin);
            
            // 计算误差
            Eigen::Matrix4d T_pred = 
                data_pin.oMf[ee_frame_id].toHomogeneousMatrix();
            Eigen::Vector3d pos_pred = T_pred.block<3,1>(0,3);
            Eigen::Matrix3d R_pred = T_pred.block<3,3>(0,0);
            
            Eigen::Vector3d pos_error = point.pos_measured - pos_pred;
            Eigen::Quaterniond quat_pred(R_pred);
            Eigen::Quaterniond quat_err = 
                point.quat_measured * quat_pred.inverse();
            Eigen::Vector3d rot_error = quat_err.vec();
            
            Eigen::VectorXd error(6);
            error.head(3) = rot_error;
            error.tail(3) = pos_error;
            
            // 计算回归矩阵
            Eigen::MatrixXd J_k(6, n_params);
            pinocchio::computeFrameKinematicRegressor(
                model, data_pin, ee_frame_id, J_k);
            
            // 堆叠
            // ...
        }
        
        // 求解参数更新
        Eigen::VectorXd delta_theta = 
            (J_all.transpose() * J_all).ldlt().solve(
                J_all.transpose() * error_all);
        
        // 更新参数
        theta += delta_theta;
        
        // 检查收敛
        if (delta_theta.norm() < 1e-6) break;
    }
    
    // 更新URDF模型
    updateURDFWithIdentifiedParameters(urdf_path, theta);
}
\end{lstlisting}

\section{双臂系统运动学辨识}

\subsection{特殊考虑}

对于双臂系统,需要辨识:

\begin{enumerate}
    \item \textbf{左臂的运动学参数}
    \item \textbf{右臂的运动学参数}
    \item \textbf{双臂之间的相对位姿}(如果双臂不是完全对称)
    \item \textbf{基座到双臂连接点的位姿}
\end{enumerate}

\subsection{数据采集策略}

\begin{lstlisting}[caption=双臂系统数据采集]
// 策略1:分别辨识每个手臂
// 固定一个手臂,只运动另一个手臂
std::vector<KinematicDataPoint> left_arm_data;
std::vector<KinematicDataPoint> right_arm_data;

// 策略2:同时辨识(考虑耦合)
// 同时运动两个手臂,测量两个末端执行器的位姿
std::vector<DualArmDataPoint> dual_arm_data;

struct DualArmDataPoint {
    std::vector<double> q_left, q_right;
    Eigen::Vector3d pos_left, pos_right;
    Eigen::Quaterniond quat_left, quat_right;
};
\end{lstlisting}

\section{全身系统运动学辨识}

\subsection{包含底盘和腰部的辨识}

对于全身系统,需要辨识:

\begin{enumerate}
    \item \textbf{底盘位姿:} 相对于世界坐标系
    \item \textbf{腰部关节参数:} 折叠和旋转关节的几何参数
    \item \textbf{双臂运动学参数}
    \item \textbf{各部分的相对位姿}
\end{enumerate}

\subsection{浮动基座的处理}

如果底盘是浮动的,需要额外考虑:

\begin{lstlisting}[caption=浮动基座运动学辨识]
// 底盘有6个自由度(位置+姿态)
// 需要测量底盘在世界坐标系中的位姿

struct FullBodyDataPoint {
    // 底盘位姿(外部测量,如IMU、视觉等)
    Eigen::Vector3d base_pos;
    Eigen::Quaterniond base_quat;
    
    // 关节角度
    std::vector<double> q_torso;  // 腰部关节
    std::vector<double> q_left;   // 左臂关节
    std::vector<double> q_right;  // 右臂关节
    
    // 末端执行器位姿(外部测量)
    Eigen::Vector3d pos_left_ee, pos_right_ee;
    Eigen::Quaterniond quat_left_ee, quat_right_ee;
};
\end{lstlisting}

\section{关节零位标定}

\subsection{零位偏置辨识}

关节零位偏置 $\theta_{i,0}$ 是重要的运动学参数,可以通过以下方法辨识:

\begin{lstlisting}[caption=关节零位标定]
// 方法1:使用已知的参考点
// 让机器人移动到已知的参考位置,测量关节角度
Eigen::VectorXd q_measured_at_reference;
// 参考位置的关节角度应该是 [0, 0, ..., 0]
// 零位偏置 = q_measured_at_reference

// 方法2:使用外部测量设备
// 测量多个已知位置的末端执行器位姿
// 通过优化求解零位偏置
void calibrateJointZeros(
    pinocchio::Model& model,
    const std::vector<ReferencePoint>& reference_points) {
    
    // 参考点:已知的末端执行器位姿
    struct ReferencePoint {
        Eigen::Vector3d pos;
        Eigen::Quaterniond quat;
        Eigen::VectorXd q_measured;  // 测量到的关节角度
    };
    
    // 优化目标:最小化预测位姿与参考位姿的误差
    // 变量:关节零位偏置
    // ...
}
\end{lstlisting}

\section{TCP 标定}

\subsection{工具中心点标定}

TCP(Tool Center Point)标定是运动学辨识的重要部分:

\begin{lstlisting}[caption=TCP标定]
// 使用 xCore SDK 的标定功能
error_code ec;
std::vector<std::array<double, 7>> calibration_points;

// 采集多个标定点
for (int i = 0; i < 6; i++) {
    // 移动到不同的姿态
    robot.MoveJ(...);
    auto q = robot.jointPos(ec);
    calibration_points.push_back(q);
}

// 执行TCP标定
auto result = robot.calibrateFrame(
    FrameType::tool, 
    calibration_points, 
    true,  // 机器人手持
    ec);

if (!ec) {
    std::cout << "TCP标定成功" << std::endl;
    std::cout << "TCP位姿: " << result.transformation << std::endl;
}
\end{lstlisting}

\section{基坐标系标定}

\subsection{基坐标系辨识}

基坐标系标定确定机器人基座相对于世界坐标系的位置和姿态:

\begin{lstlisting}[caption=基坐标系标定]
// 使用 xCore SDK 的基坐标系标定
error_code ec;
std::vector<std::array<double, 7>> calibration_points;
std::array<double, 3> base_aux = {0, 0, 0};  // 辅助点

// 采集6个标定点
for (int i = 0; i < 6; i++) {
    robot.MoveJ(...);
    auto q = robot.jointPos(ec);
    calibration_points.push_back(q);
}

// 执行基坐标系标定
auto result = robot.calibrateFrame(
    FrameType::base,
    calibration_points,
    false,  // 外部标定
    ec,
    base_aux);

if (!ec) {
    std::cout << "基坐标系标定成功" << std::endl;
    // 标定结果会自动保存到控制器
}
\end{lstlisting}

\section{实际应用中的注意事项}

\subsection{测量设备}

\begin{itemize}
    \item \textbf{视觉系统:} 使用相机和标记点(如 ArUco、AprilTag)
    \item \textbf{激光追踪器:} 高精度的位置测量(精度可达 0.01mm)
    \item \textbf{测量臂:} 便携式坐标测量机
    \item \textbf{IMU:} 用于测量姿态(特别是浮动基座)
\end{itemize}

\subsection{数据同步}

\begin{lstlisting}[caption=数据同步]
// 确保关节角度和外部测量数据时间同步
struct SynchronizedData {
    double timestamp;
    std::vector<double> q;
    Eigen::Vector3d pos_external;
    Eigen::Quaterniond quat_external;
};

// 使用时间戳对齐数据
void synchronizeData(
    const std::vector<JointData>& joint_data,
    const std::vector<ExternalMeasurement>& external_data) {
    // 根据时间戳匹配数据
    // ...
}
\end{lstlisting}

\subsection{误差权重}

由于位置和姿态的误差量级不同,需要设置权重:

\begin{equation}
\boldsymbol{\epsilon}_w = \mathbf{W} \boldsymbol{\epsilon}
\end{equation}

其中 $\mathbf{W}$ 是权重矩阵:

\begin{equation}
\mathbf{W} = \begin{bmatrix}
    w_r \mathbf{I}_3 & \mathbf{0} \\
    \mathbf{0} & w_p \mathbf{I}_3
\end{bmatrix}
\end{equation}

通常 $w_r \approx 1$(姿态),$w_p \approx 100-1000$(位置,单位:m$^{-1}$)。

\section{完整辨识流程示例}

\begin{algorithm}
\caption{运动学参数辨识完整流程}
\begin{algorithmic}[1]
\REQUIRE URDF模型文件、外部测量设备、数据采集系统
\ENSURE 辨识的运动学参数
\STATE 加载URDF模型,初始化Pinocchio
\STATE 设计标定轨迹(覆盖工作空间)
\STATE \textbf{for} 每个标定点 \textbf{do}
    \STATE 控制机器人移动到目标位置
    \STATE 读取关节角度 q
    \STATE 使用外部设备测量末端执行器位姿
    \STATE 保存数据点 (q, pos\_measured, quat\_measured)
\STATE \textbf{end for}
\STATE 初始化运动学参数(使用URDF中的值)
\STATE \textbf{repeat}
    \STATE 对每个数据点计算预测位姿和误差
    \STATE 计算运动学回归矩阵
    \STATE 求解参数更新:$\delta \theta = (J^T J)^{-1} J^T \epsilon$
    \STATE 更新参数:$\theta = \theta + \delta \theta$
    \STATE 更新模型参数
\STATE \textbf{until} 收敛或达到最大迭代次数
\STATE 使用验证数据评估辨识精度
\STATE \textbf{if} 精度不满足要求 \textbf{then}
    \STATE 增加标定点,重新采集数据
\STATE \textbf{end if}
\STATE 更新URDF模型中的参数
\end{algorithmic}
\end{algorithm}

\section{与动力学辨识的联合}

\subsection{联合辨识}

运动学和动力学参数可以联合辨识:

\begin{enumerate}
    \item \textbf{先进行运动学辨识:} 使用外部测量设备辨识几何参数
    \item \textbf{再进行动力学辨识:} 使用辨识后的运动学参数,辨识动力学参数
    \item \textbf{迭代优化:} 可以交替优化运动学和动力学参数
\end{enumerate}

\subsection{优势}

\begin{itemize}
    \item 运动学参数准确后,动力学辨识更准确
    \item 可以同时优化所有参数
    \item 提高整体模型精度
\end{itemize}

\section{总结}

\begin{itemize}
    \item \textbf{Pinocchio 支持运动学辨识:} 提供了回归矩阵计算、正向运动学等关键功能
    \item \textbf{需要外部测量:} 运动学辨识通常需要外部测量设备(视觉、激光追踪器等)
    \item \textbf{非线性优化:} 由于运动学方程非线性,需要迭代优化
    \item \textbf{适用于复杂系统:} 单臂、双臂、全身系统都可以使用相同的方法
    \item \textbf{与SDK集成:} xCore SDK 提供了TCP和基坐标系标定功能
    \item \textbf{联合辨识:} 可以与动力学辨识联合进行,提高整体精度
\end{itemize}

\section{参考文献}

\begin{enumerate}
    \item Pinocchio Documentation: \url{https://stack-of-tasks.github.io/pinocchio/}
    \item Hollerbach, J. M., and Wampler, C. W. (1996). "The calibration index and taxonomy for robot kinematic calibration methods." \textit{The International Journal of Robotics Research}, 15(6), 573-591.
    \item Chen, I. M., and Yang, G. (1999). "Kinematic calibration of modular reconfigurable robots using product-of-exponentials formula." \textit{Journal of Robotic Systems}, 16(11), 587-597.
    \item Hayati, S., and Mirmirani, M. (1985). "Improving the absolute positioning accuracy of robot manipulators." \textit{Journal of Robotic Systems}, 2(4), 397-413.
    \item \texttt{hybrid\_force\_position\_control.cpp} - ROKAE xCore SDK 示例代码
    \item xCore SDK 文档:坐标系标定功能
\end{enumerate}

\end{document}

