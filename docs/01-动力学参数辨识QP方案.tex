\documentclass[12pt,a4paper]{article}
\usepackage[UTF8]{ctex}
\usepackage{amsmath}
\usepackage{amssymb}
\usepackage{geometry}
\usepackage{enumitem}
\usepackage{hyperref}
\usepackage{bm}

\geometry{margin=2.5cm}

\title{动力学参数辨识的带约束优化(QP)方案:理论详述}
\author{Dynamics\_identification\_Frank}
\date{\today}

\begin{document}

\maketitle

\begin{abstract}
本文从理论上详述基于二次规划(QP)的机器人动力学参数辨识方法。在回归模型 $\bm{\tau} = \bm{Y}\bm{\theta}$ 的基础上,将“质量为正、惯性张量对称正定且满足惯性椭球三角不等式”等物理约束显式纳入优化问题,得到在可行域内最小化拟合误差的辨识解,并与无约束最小二乘加后处理投影的方案进行对比。
\end{abstract}

\tableofcontents
\newpage

%==============================================================================
\section{问题背景与回归模型}
%==============================================================================

\subsection{线性参数化动力学}

机器人刚体动力学可写为关于惯性参数的线性形式(最小参数集):
\begin{equation}
  \bm{\tau} = \bm{Y}(\bm{q},\dot{\bm{q}},\ddot{\bm{q}})\, \bm{\theta}
  \label{eq:regressor}
\end{equation}
其中 $\bm{\tau}\in\mathbb{R}^n$ 为关节力矩,$\bm{Y}\in\mathbb{R}^{n\times p}$ 为回归矩阵,$\bm{\theta}\in\mathbb{R}^p$ 为动力学参数向量。对 $n_q=7$ 的机械臂采用每连杆 10 维最小惯性参数时,$p = 7\times 10 = 70$。

\subsection{批量测量与最小二乘}

设采集 $N$ 个时刻的数据,堆叠为:
\begin{equation}
  \bm{\tau}_{\mathrm{all}} = \bm{Y}_{\mathrm{all}}\, \bm{\theta},\qquad
  \bm{Y}_{\mathrm{all}} \in \mathbb{R}^{(nN)\times p},\quad
  \bm{\tau}_{\mathrm{all}} \in \mathbb{R}^{nN}.
\end{equation}
无约束最小二乘解为
\begin{equation}
  \min_{\bm{\theta}}\; \|\bm{Y}_{\mathrm{all}}\bm{\theta} - \bm{\tau}_{\mathrm{all}}\|_2^2
  \quad\Rightarrow\quad
  \bm{\theta}_{\mathrm{LS}} = (\bm{Y}_{\mathrm{all}}^\top \bm{Y}_{\mathrm{all}})^{\dagger}
  \bm{Y}_{\mathrm{all}}^\top \bm{\tau}_{\mathrm{all}},
\end{equation}
或带 URDF 先验的 Ridge 解:
\begin{equation}
  \min_{\bm{\theta}}\; \|\bm{Y}_{\mathrm{all}}\bm{\theta} - \bm{\tau}_{\mathrm{all}}\|_2^2
  + \lambda \|\bm{\theta} - \bm{\theta}_{\mathrm{urdf}}\|_2^2.
\end{equation}
当 $\bm{Y}_{\mathrm{all}}$ 不满秩或病态时,$\bm{\theta}_{\mathrm{LS}}$(或 Ridge 解)可能对应非物理惯性(负质量、非正定惯性张量),导致写进 URDF 后质量矩阵 $\bm{M}(\bm{q})$ 不正定,仿真或控制出错。

%==============================================================================
\section{物理约束的数学表述}
%==============================================================================

将 $\bm{\theta}$ 按连杆分块,第 $i$ 个连杆对应 10 维子向量 $\bm{\pi}_i\in\mathbb{R}^{10}$,通过 Pinocchio 的 \texttt{Inertia::FromDynamicParameters} 映射为质量 $m_i$、质心 $\bm{c}_i\in\mathbb{R}^3$、惯性张量 $\bm{I}_i\in\mathbb{R}^{3\times 3}$(对称)。物理可行性要求如下。

\subsection{质量约束}

每个连杆质量为正,且通常不低于一给定下限(数值稳定性或工艺下限):
\begin{equation}
  m_i \geq m_{\min} > 0,\qquad i=1,\ldots,n_{\mathrm{links}}.
\end{equation}
在最小参数集中,$m_i$ 由 $\bm{\pi}_i$ 的第一维给出,故这是关于 $\bm{\theta}$ 的线性不等式约束(每个连杆一条)。

\subsection{惯性张量对称正定}

惯性张量在连杆局部坐标系下必须对称正定:
\begin{equation}
  \bm{I}_i = \bm{I}_i^\top,\qquad \bm{I}_i \succ 0.
\end{equation}
对称性由参数化自动保证;$\bm{I}_i\succ 0$ 等价于特征值全为正:
\begin{equation}
  \lambda_{i,1},\, \lambda_{i,2},\, \lambda_{i,3} > 0.
\end{equation}
等价地,所有顺序主子式为正,或存在 Cholesky 分解 $\bm{I}_i = \bm{L}_i\bm{L}_i^\top$($\bm{L}_i$ 下三角且对角元为正)。正定性关于 $\bm{\pi}_i$ 是非线性的($\bm{\pi}_i \mapsto \bm{I}_i$ 为非线性映射),因此约束在参数空间 $\bm{\theta}$ 上为非线性。

\subsection{惯性椭球的三角不等式}

刚体惯性张量的三个主惯性矩(即 $\bm{I}_i$ 的特征值)除为正外,还满足“三角不等式”:
\begin{equation}
  \lambda_{i,a} + \lambda_{i,b} \geq \lambda_{i,c}
  \quad\text{($a,b,c$ 为 $1,2,3$ 的任意排列)}.
\end{equation}
例如 $\lambda_1 \leq \lambda_2 \leq \lambda_3$ 时,需
\begin{equation}
  \lambda_1 + \lambda_2 \geq \lambda_3.
\end{equation}
该约束与 $\bm{I}_i\succ 0$ 一起,构成“物理可实现惯性张量”的完整条件。

%==============================================================================
\section{带约束优化的目标与决策变量}
%==============================================================================

\subsection{决策变量两种取法}

\begin{enumerate}[label=(\alph*)]
  \item \textbf{直接用 $\bm{\theta}$}:决策变量为 $\bm{\theta}\in\mathbb{R}^p$。目标与 Ridge 一致;约束需用 $\bm{\theta}$ 表示:$m_i(\bm{\pi}_i)\geq m_{\min}$ 为线性,$\bm{I}_i(\bm{\pi}_i)\succ 0$ 及三角不等式为非线性(且 $\bm{\pi}_i\mapsto\bm{I}_i$ 的显式形式较复杂),适合在求解器中用黑箱约束或逐次线性化/凸近似。
  \item \textbf{用每连杆 $(m_i,\bm{c}_i,\bm{I}_i)$}:决策变量为各连杆的质量、质心、对称惯性矩阵,再通过最小参数集与 $\bm{\theta}$ 的对应关系写回 $\bm{\theta}$,或直接在回归的“物理参数”形式下建模型。此时 $m_i\geq m_{\min}$ 为线性,$\bm{I}_i\succ 0$ 可用 Cholesky 或半定约束表述,三角不等式为特征值或对称矩阵元素的线性/二阶锥约束。表述更直观,但变量与回归矩阵的耦合需在优化中处理(例如用 $\bm{\theta}=\bm{\phi}(m_1,\bm{c}_1,\bm{I}_1,\ldots)$ 代入目标)。
\end{enumerate}

以下以决策变量为 $\bm{\theta}$ 为主,约束在“$\bm{\theta}$ 空间”中表述(便于与现有 Ridge 流程一致);等价地可改写为在 $(m,\bm{c},\bm{I})$ 空间中的约束。

\subsection{目标函数}

带正则的二次目标(与 Ridge 一致):
\begin{equation}
  J(\bm{\theta}) = \|\bm{Y}_{\mathrm{all}}\bm{\theta} - \bm{\tau}_{\mathrm{all}}\|_2^2
  + \lambda\,\|\bm{\theta} - \bm{\theta}_{\mathrm{urdf}}\|_2^2.
\end{equation}
展开为标准的 QP 形式:
\begin{equation}
  J(\bm{\theta}) = \frac{1}{2}\bm{\theta}^\top \bm{H}\,\bm{\theta} + \bm{g}^\top\bm{\theta} + \mathrm{const},
\end{equation}
其中
\begin{equation}
  \bm{H} = 2\bigl(\bm{Y}_{\mathrm{all}}^\top \bm{Y}_{\mathrm{all}} + \lambda\bm{I}\bigr),\qquad
  \bm{g} = -2\bigl(\bm{Y}_{\mathrm{all}}^\top \bm{\tau}_{\mathrm{all}} + \lambda\bm{\theta}_{\mathrm{urdf}}\bigr).
\end{equation}
$\bm{H}\succeq 0$,无约束时该 QP 有唯一最小解(当 $\lambda>0$ 时 $\bm{H}\succ 0$)。

%==============================================================================
\section{约束的表述方式(用于 QP/SOCP/SDP)}
%==============================================================================

\subsection{质量约束}

设 $\bm{\theta}$ 中与第 $i$ 个连杆质量对应的分量为 $\theta_{k_i}$(在 10 维块中的第一维),则
\begin{equation}
  \theta_{k_i} \geq m_{\min},\qquad i=1,\ldots,n_{\mathrm{links}}.
\end{equation}
这是关于 $\bm{\theta}$ 的线性不等式,可直接用于 QP 的 $\bm{A}\bm{\theta}\leq \bm{b}$ 或 $\bm{A}_{\mathrm{eq}}\bm{\theta}=\bm{b}_{\mathrm{eq}}$ 的等价形式(如 $-\theta_{k_i}\leq -m_{\min}$)。

\subsection{惯性张量正定:半定规划(SDP)表述}

对第 $i$ 个连杆,要求 $\bm{I}_i \succeq \delta\bm{I}$($\delta>0$ 为小常数,保证正定)。$\bm{I}_i$ 由 $\bm{\pi}_i$ 唯一确定,记作 $\bm{I}_i(\bm{\pi}_i)$。则约束为
\begin{equation}
  \bm{I}_i(\bm{\pi}_i) - \delta\bm{I} \succeq 0.
\end{equation}
这是关于 $\bm{\pi}_i$ 的线性矩阵不等式(LMI):$\bm{I}_i$ 的 6 个独立分量是 $\bm{\pi}_i$ 中若干分量的线性函数,因此 LMI 关于 $\bm{\pi}_i$ 是线性的,关于 $\bm{\theta}$ 也是线性的。整体问题变为:二次目标 + 线性不等式(质量)+ 多个 LMI(每连杆一个),即 \textbf{半定规划(SDP)},可用 Mosek、SeDuMi、SDPT3 等求解。

\subsection{惯性张量正定:Cholesky 与二阶锥(SOC)表述}

$\bm{I}_i \succ 0$ 等价于存在下三角 $\bm{L}_i$ 且对角元 $L_{jj}>0$,使得 $\bm{I}_i = \bm{L}_i\bm{L}_i^\top$。将 $\bm{L}_i$ 的元素作为辅助变量,约束 $\bm{I}_i = \bm{L}_i\bm{L}_i^\top$ 为关于 $\bm{L}_i$ 与 $\bm{I}_i$ 的二次等式;再对 $\bm{\pi}_i$ 与 $\bm{I}_i$ 的关系(线性)做等式约束。对角元 $L_{jj}>0$ 可写为 $L_{jj}\geq \epsilon$。此类约束可进一步用二阶锥(SOC)或线性化手段嵌入 SOCP,从而用 OSQP(若可写为 QP+锥)、ECOS、Mosek 等求解。

\subsection{惯性三角不等式的线性表述}

在主惯性矩 $\lambda_1\leq \lambda_2\leq \lambda_3$ 下,三角不等式为 $\lambda_1+\lambda_2 - \lambda_3 \geq 0$。若在优化中显式引入特征值变量 $\lambda_{i,1},\lambda_{i,2},\lambda_{i,3}$ 及特征向量,则需满足 $\bm{I}_i = \bm{V}\bm{\Lambda}\bm{V}^\top$、$\bm{V}^\top\bm{V}=\bm{I}$,约束非凸且变量增多。更实用的做法是:
\begin{itemize}
  \item 在 \textbf{SDP 表述}中,仅约束 $\bm{I}_i \succeq \delta\bm{I}$;物理上“可实现”的惯性张量自动满足三角不等式(正定对称 3\times 3 矩阵若来自刚体惯性,则必满足),因此若回归/数据导致不满足,可视为模型或数据问题,或
  \item 在 \textbf{后处理投影}中显式强制三角不等式(如对特征值裁剪后再组合),而不在 QP/SDP 中写。
\end{itemize}
若坚持在优化中写三角不等式,可在每连杆上引入 $\lambda$ 与 $\bm{I}_i$ 的等式 $\bm{I}_i\bm{v}_k = \lambda_k\bm{v}_k$,再加 $\lambda_1+\lambda_2\geq \lambda_3$ 等,问题会变为非凸,需用非线性规划或序列凸近似。

%==============================================================================
\section{标准 QP/SDP 形式小结}
%==============================================================================

\subsection{仅质量约束(线性不等式)}

若暂时只保留 $m_i\geq m_{\min}$,则问题为
\begin{align}
  \min_{\bm{\theta}}\quad & \frac{1}{2}\bm{\theta}^\top \bm{H}\,\bm{\theta} + \bm{g}^\top\bm{\theta} \\
  \text{s.t.}\quad & \bm{A}\,\bm{\theta} \leq \bm{b}
\end{align}
(将 $\theta_{k_i}\geq m_{\min}$ 写为 $-\theta_{k_i}\leq -m_{\min}$ 放入 $\bm{A},\bm{b}$)。这是 \textbf{标准 QP},可用 OSQP、qpOASES、QuadProg 等求解,实现简单、计算高效。

\subsection{质量 + 惯性正定(LMI)}

在质量约束基础上,对每连杆增加 $\bm{I}_i(\bm{\pi}_i) - \delta\bm{I} \succeq 0$。目标仍为 $\bm{\theta}$ 的二次函数,约束为线性不等式 + 多个 LMI,即 \textbf{SDP}:
\begin{align}
  \min_{\bm{\theta}}\quad & \frac{1}{2}\bm{\theta}^\top \bm{H}\,\bm{\theta} + \bm{g}^\top\bm{\theta} \\
  \text{s.t.}\quad & \bm{A}\,\bm{\theta} \leq \bm{b}, \\
  & \bm{I}_i(\bm{\pi}_i) \succeq \delta\bm{I},\quad i=1,\ldots,n_{\mathrm{links}}.
\end{align}
需将 $\bm{I}_i(\bm{\pi}_i)$ 用 $\bm{\pi}_i$(即 $\bm{\theta}$ 的相应分量)的线性函数写出(Pinocchio 的 10 维参数与 $m,\bm{c},\bm{I}$ 的对应关系为线性),即可用 SDP 求解器求解。

\subsection{与“无约束 + 投影”的对比}

\begin{itemize}
  \item \textbf{无约束 Ridge + 投影}:先求无约束 Ridge 解 $\bm{\theta}^*$,再对每连杆 $(m_i,\bm{c}_i,\bm{I}_i)$ 做投影($m_i\leftarrow\max(m_i,m_{\min})$,$\bm{I}_i$ 做最近正定与三角不等式修正),写回 $\bm{\theta}$。优点是不需要 QP/SDP 求解器,实现简单;缺点是投影后的 $\bm{\theta}$ 不再是最小化 $J(\bm{\theta})$ 的解,拟合误差会略大。
  \item \textbf{QP/SDP}:在可行域($m_i\geq m_{\min}$,$\bm{I}_i\succeq \delta\bm{I}$)内直接最小化 $J(\bm{\theta})$,得到的是“物理可行且拟合最优”的解;缺点是需要引入求解器并写清 $\bm{I}_i(\bm{\pi}_i)$ 的线性表达式或 LMI 接口。
\end{itemize}

%==============================================================================
\section{实现要点与推荐流程}
%==============================================================================

\begin{enumerate}
  \item \textbf{回归矩阵与数据}:与现有 step2 一致,用 Pinocchio 计算 $\bm{Y}_{\mathrm{all}}$,采集 $\bm{\tau}_{\mathrm{all}}$。
  \item \textbf{URDF 先验}:从参考 URDF 得到 $\bm{\theta}_{\mathrm{urdf}}$,用于 Ridge 项 $\lambda\|\bm{\theta}-\bm{\theta}_{\mathrm{urdf}}\|^2$。
  \item \textbf{质量约束}:从 Pinocchio 的 10 维参数顺序确定每块第一维对应质量,写出 $\bm{A},\bm{b}$。
  \item \textbf{惯性 LMI}:对每个 10 维块,用 \texttt{Inertia::FromDynamicParameters} 的线性关系写出 $\bm{I}_i$ 的 6 个独立元关于 $\bm{\pi}_i$ 的线性表达式,再转为 SDP 的 LMI 格式(或先用“仅质量约束”的 QP 做简化版)。
  \item \textbf{求解}:调用 QP 求解器(仅质量)或 SDP 求解器(质量+惯性正定),得到 $\bm{\theta}_{\mathrm{QP}}$;若未在优化中加三角不等式,可对结果再做一次三角不等式投影后写 URDF。
\end{enumerate}

%==============================================================================
\section{参考文献与工具}
%==============================================================================

\begin{itemize}
  \item Pinocchio: \url{https://stack-of-tasks.github.io/pinocchio/}
  \item OSQP (QP): \url{https://osqp.org/}
  \item Mosek (QP/SDP/SOCP): \url{https://www.mosek.com/}
  \item 惯性参数物理条件:刚体惯性张量正定与三角不等式的标准结论见机器人学教材(如 Khalil, Spong 等)的动力学参数辨识章节。
\end{itemize}

\end{document}
