\documentclass[12pt,a4paper]{article}
\usepackage[UTF8]{ctex}
\usepackage{amsmath}
\usepackage{amssymb}
\usepackage{geometry}
\usepackage{enumitem}
\usepackage{hyperref}
\usepackage{bm}
\usepackage{listings}
\usepackage{xcolor}

\geometry{margin=2.5cm}

\title{step2\_dynamics\_parameter\_estimation 程序运行说明}
\author{Dynamics\_identification\_Frank}
\date{\today}

\begin{document}

\maketitle

\begin{abstract}
本文详细说明 \texttt{step2\_dynamics\_parameter\_estimation.cpp} 的运行流程,包括配置加载、数据与 URDF 读入、回归矩阵构造、Ridge/QP 求解、物理约束投影及各类输出文件的含义。重点说明 \texttt{dynamics\_parameters\_urdf.csv} 的来源与生成方式。
\end{abstract}

\tableofcontents
\newpage

%==============================================================================
\section{程序入口与配置加载}
%==============================================================================

\subsection{main 函数流程(约 1033--1081 行)}

程序从 \texttt{main} 进入后依次执行:

\begin{enumerate}[leftmargin=*]
  \item \textbf{查找配置文件}:依次尝试
    \begin{itemize}
      \item \texttt{config/step2\_dynamics\_parameter\_estimation.yaml}(相对当前工作目录)
      \item \texttt{src/config/step2\_dynamics\_parameter\_estimation.yaml}
      \item \texttt{<可执行文件所在目录>/config/step2\_dynamics\_parameter\_estimation.yaml}
    \end{itemize}
    找到第一个存在的路径即停止。
  \item \textbf{加载配置}:调用 \texttt{loadStep2Config(config\_path)},按行解析 YAML(\texttt{key: value},忽略 \texttt{\#} 注释),填充 \texttt{Step2Config} 结构体,得到数据文件路径、URDF 路径、\texttt{lambda\_rel}、\texttt{m\_min}、\texttt{I\_eps}、\texttt{I\_trace\_min} 以及各输出文件名(含 \texttt{dynamics\_parameters\_urdf\_csv})。
  \item \textbf{命令行覆盖}:若 \texttt{argc>1},用 \texttt{argv[1]} 覆盖配置中的 \texttt{data\_file};若 \texttt{argc>2},用 \texttt{argv[2]} 覆盖 \texttt{urdf\_file}。
  \item \textbf{调用辨识主函数}:\texttt{estimateDynamicsParameters(data\_file, urdf\_file, cfg)}。
\end{enumerate}

\subsection{Step2Config 中的相关字段}

与 \texttt{dynamics\_parameters\_urdf.csv} 直接相关的是:
\begin{itemize}
  \item \texttt{dynamics\_parameters\_urdf\_csv}:默认 \texttt{"dynamics\_parameters\_urdf.csv"},即该 CSV 的写出路径(相对当前工作目录或绝对路径)。
\end{itemize}

%==============================================================================
\section{estimateDynamicsParameters 主流程}
%==============================================================================

\subsection{1. Pinocchio 模型与重力(约 506--530 行)}

\begin{itemize}
  \item 使用 \texttt{pinocchio::urdf::buildModel(urdf\_file, pinocchio\_model)} 从传入的 \textbf{URDF 文件} 构建 \texttt{pinocchio\_model}。
  \item 设置基座到世界的旋转与重力方向,并构造 \texttt{pinocchio\_data}。
\end{itemize}

\textbf{重要}:此后所有“来自 URDF”的惯性信息都来自该 \texttt{pinocchio\_model},即本次运行时的 \texttt{urdf\_file}(配置或命令行)。

\subsection{2. 加载采集数据(约 536--543 行)}

调用 \texttt{loadDataFromCSV(data\_file)},按行解析 CSV,得到 \texttt{collected\_data}:每行对应一个 \texttt{DataPoint},包含 \texttt{time, q0..q6, dq0..dq6, ddq0..ddq6, tau0..tau6}。

\subsection{3. 确定参数维数并构造 \texttt{theta\_urdf}(约 548--585 行)}

\begin{enumerate}[leftmargin=*]
  \item 用一组零状态 \texttt{(q\_test, dq\_test, ddq\_test)} 调用 \texttt{pinocchio::computeJointTorqueRegressor(...)},得到回归矩阵 \texttt{Y\_test},其列数 \texttt{n\_params = Y\_test.cols()}(7 自由度、每连杆 10 维时为 70)。
  \item 若 \texttt{n\_params == 10*(njoints-1)}(例如 70),则从 \textbf{当前已加载的 URDF 模型} 构造 URDF 先验向量 \texttt{theta\_urdf}:
\end{enumerate}

\begin{lstlisting}[language=C++,basicstyle=\ttfamily\small,keywordstyle=\color{blue}]
theta_urdf.resize(n_params);
theta_urdf.setZero();
for (pinocchio::JointIndex jid = 1; jid < pinocchio_model.njoints; ++jid) {
  const auto pi = pinocchio_model.inertias[jid].toDynamicParameters(); // 10x1
  const int base = 10 * (jid - 1);
  theta_urdf.segment(base, 10) = pi;
}
\end{lstlisting}

含义:
\begin{itemize}
  \item \texttt{pinocchio\_model.inertias[jid]} 是关节 \texttt{jid} 对应连杆的惯性(质量、质心、惯性张量),由 Pinocchio 在 \texttt{buildModel(urdf\_file, ...)} 时从 URDF 解析得到。
  \item \texttt{toDynamicParameters()} 将该惯性转为 Pinocchio 约定的 10 维向量 $\bm{\pi}$,顺序为
    \[
    [m,\; mc_x,\; mc_y,\; mc_z,\; I_{xx},\; I_{xy},\; I_{yy},\; I_{xz},\; I_{yz},\; I_{zz}]^\top
    \]
    (质心处惯性,与回归矩阵列顺序一致)。
  \item \texttt{theta\_urdf} 按关节索引依次堆叠:\texttt{theta\_urdf[0:9]} 为连杆 1,\texttt{theta\_urdf[10:19]} 为连杆 2,……,共 70 维。
\end{itemize}

因此,\textbf{\texttt{theta\_urdf} 完全由“本次运行所加载的 URDF”决定},与采集数据、求解方式无关。

\subsection{4. 构造全局回归矩阵与力矩向量(约 597--675 行)}

对每个 \texttt{DataPoint}:
\begin{itemize}
  \item 用 \texttt{pinocchio::computeJointTorqueRegressor(pinocchio\_model, pinocchio\_data, q, dq, ddq)} 得到该时刻的 $7\times 70$ 回归矩阵 \texttt{Y};
  \item 将 \texttt{Y} 按行拼到 \texttt{Y\_all},将 \texttt{tau} 拼到 \texttt{tau\_all}。
\end{itemize}

得到 $\bm{Y}_{\mathrm{all}}\bm{\theta}=\bm{\tau}_{\mathrm{all}}$,其中 $\bm{Y}_{\mathrm{all}}\in\mathbb{R}^{(7N)\times 70}$,$N$ 为数据点数。

\subsection{5. 求解与投影(约 676--758 行)}

\begin{itemize}
  \item 用 SVD 求最小范数最小二乘解 \texttt{theta\_ls};
  \item 若存在 \texttt{theta\_urdf},则用 Ridge 或 QP(OSQP,带质量下界 \texttt{m\_min})求
    \[
    \min_{\bm{\theta}}\; \|\bm{Y}_{\mathrm{all}}\bm{\theta}-\bm{\tau}_{\mathrm{all}}\|^2 + \lambda\|\bm{\theta}-\bm{\theta}_{\mathrm{urdf}}\|^2,\quad \text{并满足质量等约束},
    \]
    得到 \texttt{theta\_estimated};
  \item 对 \texttt{theta\_estimated} 做物理约束投影 \texttt{projectThetaToPhysical(...)}:质量 $\ge$ \texttt{m\_min},惯性对称正定且满足三角不等式;若某连杆投影后 $\mathrm{trace}(\bm{I})<\texttt{I\_trace\_min}$,则用 \texttt{theta\_urdf} 中该连杆的 10 维替换(避免写出 1e-6 量级无效惯量)。
\end{itemize}

\subsection{6. 写出各输出文件(约 753--871 行)}

\begin{itemize}
  \item \texttt{cfg.result\_file}:辨识结果摘要与全部 \texttt{theta\_estimated};
  \item \texttt{cfg.dynamics\_parameters\_csv}:\textbf{投影后的辨识参数} \texttt{theta\_estimated}(\texttt{parameter\_index, value});
  \item \texttt{cfg.dynamics\_parameters\_ls\_csv}:Ridge 最小二乘解 \texttt{theta\_ls\_ridge}(无投影);
  \item \textbf{\texttt{cfg.dynamics\_parameters\_urdf\_csv}(即 \texttt{dynamics\_parameters\_urdf.csv})}:见下节;
  \item \texttt{cfg.dynamics\_physical\_parameters\_txt}:由 \texttt{theta\_estimated} 经 \texttt{FromDynamicParameters} 还原的质量/质心/惯性,与 URDF 对比;
  \item \texttt{cfg.output\_urdf}:用 \texttt{theta\_estimated} 写回的辨识后 URDF。
\end{itemize}

%==============================================================================
\section{dynamics\_parameters\_urdf.csv 的生成方式}
%==============================================================================

\subsection{写出条件与代码位置}

仅当 \texttt{theta\_urdf.size() == n\_params}(即已成功从 URDF 构造 70 维先验)时才会写该文件。代码位于约 861--871 行:

\begin{lstlisting}[language=C++,basicstyle=\ttfamily\small]
if (theta_urdf.size() == n_params) {
  std::ofstream param_file_urdf(cfg.dynamics_parameters_urdf_csv);
  param_file_urdf << "parameter_index,value\n";
  for (int i = 0; i < theta_urdf.size(); i++) {
    param_file_urdf << i << "," << std::scientific << std::setprecision(10)
                    << theta_urdf(i) << "\n";
  }
  param_file_urdf.close();
}
\end{lstlisting}

即:将向量 \texttt{theta\_urdf} 按 \texttt{parameter\_index = 0,1,...,69} 逐行写出为 \texttt{parameter\_index,value} 两列 CSV。

\subsection{theta\_urdf 的来源(与 URDF 的对应关系)}

\begin{enumerate}[leftmargin=*]
  \item \texttt{theta\_urdf} 在“构造回归矩阵”阶段(约 568--584 行)由 \textbf{当前加载的 Pinocchio 模型} 一次性构造,且之后不再修改。
  \item 该模型来自 \texttt{pinocchio::urdf::buildModel(urdf\_file, pinocchio\_model)},即 \textbf{本次运行时的 \texttt{urdf\_file}}(若未用命令行覆盖,则为配置中的 \texttt{urdf:})。
  \item 对每个关节 \texttt{jid = 1..7},\texttt{theta\_urdf.segment(10*(jid-1), 10)} 等于 \texttt{pinocchio\_model.inertias[jid].toDynamicParameters()},即该连杆在 URDF 中的惯性经 Pinocchio 转为 10 维动力学参数。
\end{enumerate}

因此:\textbf{\texttt{dynamics\_parameters\_urdf.csv} 中 70 个数值,就是“本次运行所用 URDF”中 7 个连杆的惯性,按 Pinocchio 的 10 维顺序依次堆叠后写出的结果};与采集数据、Ridge/QP 求解、投影无关。

\subsection{10 维顺序与 URDF 标签的对应}

每连杆 10 维与 URDF 的对应关系为(参见 \texttt{docs/dynamic\_parameters\_10d\_conversion.md}):
\begin{center}
\begin{tabular}{|c|l|l|}
\hline
\texttt{parameter\_index} & 含义 & URDF 对应 \\
\hline
0 & 质量 $m$ & \texttt{<mass value="..."/>} \\
1,2,3 & $m c_x,\, m c_y,\, m c_z$ & \texttt{<origin xyz="cx cy cz"/>}(质心) \\
4--9 & $I_{xx}, I_{xy}, I_{yy}, I_{xz}, I_{yz}, I_{zz}$(质心处) & \texttt{<inertia ixx ixy iyy ixz iyz izz/>} \\
\hline
\end{tabular}
\end{center}

索引 0--9 对应连杆 1,10--19 对应连杆 2,依此类推。

%==============================================================================
\section{小结}
%==============================================================================

\begin{itemize}
  \item \texttt{step2\_dynamics\_parameter\_estimation} 的运行顺序为:加载配置与命令行 $\to$ 从 URDF 建 Pinocchio 模型 $\to$ 从 CSV 读采集数据 $\to$ 用 URDF 模型构造 \texttt{theta\_urdf} 并构造回归矩阵 \texttt{Y\_all}、\texttt{tau\_all} $\to$ Ridge/QP 求解得 \texttt{theta\_estimated} $\to$ 物理投影 $\to$ 写出多种 CSV 与 URDF。
  \item \texttt{dynamics\_parameters\_urdf.csv} 的内容即为 \texttt{theta\_urdf}:由 \textbf{本次运行所加载的 URDF} 经 \texttt{pinocchio\_model.inertias[jid].toDynamicParameters()} 得到,按 \texttt{parameter\_index} 0--69 写出,与辨识结果、投影无关;用于对比或作为 Ridge 先验时的参考。
\end{itemize}

\end{document}
