\documentclass[12pt,a4paper]{article}
\usepackage[UTF8]{ctex}
\usepackage{amsmath}
\usepackage{amssymb}
\usepackage{geometry}
\usepackage{enumitem}
\usepackage{hyperref}
\usepackage{bm}
\usepackage{listings}
\usepackage{xcolor}
\usepackage{booktabs}

\geometry{margin=2.5cm}

\title{惯性参考点与质心惯量:代码中的转换关系}
\author{Dynamics\_identification\_Frank}
\date{\today}

\begin{document}

\maketitle

\begin{abstract}
本文详细说明 \texttt{step2\_dynamics\_parameter\_estimation.cpp} 中“连杆原点处转动惯量”$I_{\mathrm{origin}}$ 与“质心处转动惯量”$I_C$ 的区分与转换。程序与 Pinocchio、URDF、10 维动力学参数之间的约定一致:对外(CSV、URDF 文件、报告)统一使用质心处惯量;Pinocchio 内部及物理投影的中间步骤使用连杆原点处惯量,并通过平行轴定理正确转换。
\end{abstract}

\tableofcontents
\newpage

%==============================================================================
\section{概念与公式}
%==============================================================================

\subsection{两个参考点}

对每个连杆,惯性张量依赖于参考点与坐标系:

\begin{itemize}[leftmargin=*]
  \item \textbf{连杆原点(参考点)}:连杆坐标系原点,即 Pinocchio 中该 body 的附着点。记该点处的 $3\times 3$ 惯性张量为 $I_{\mathrm{origin}}$。
  \item \textbf{质心}:质心在连杆坐标系下的位置为 $\bm{c}=(c_x,c_y,c_z)^\top$。记质心处的惯性张量为 $I_C$。
\end{itemize}

\subsection{平行轴定理}

两者满足(Pinocchio 与多数动力学库采用此约定):
\begin{equation}
  I_{\mathrm{origin}} = I_C + m\,S(\bm{c})^\top S(\bm{c})
  \label{eq:parallel}
\end{equation}
其中 $S(\bm{c})$ 为 $\bm{c}$ 的叉积反对称矩阵:$S(\bm{c})\bm{v}=\bm{c}\times\bm{v}$。因此
\begin{equation}
  I_C = I_{\mathrm{origin}} - m\,S(\bm{c})^\top S(\bm{c}).
  \label{eq:origin_to_com}
\end{equation}

在 Pinocchio 源码中,$m\,S^\top S$ 对应 \texttt{AlphaSkewSquare(mass(), lever())}(\texttt{pinocchio/spatial/symmetric3.hpp})。

\subsection{10 维动力学参数与 URDF 约定}

\begin{itemize}[leftmargin=*]
  \item \textbf{10 维向量}(每连杆):
    \[
    \bm{\pi} = [m,\; mc_x,\; mc_y,\; mc_z,\; I_{xx},\; I_{xy},\; I_{yy},\; I_{xz},\; I_{yz},\; I_{zz}]^\top.
    \]
    其中后 6 个为 \textbf{质心处}惯性 $I_C$ 的 6 个独立分量(顺序与 Pinocchio 的 \texttt{Symmetric3} 一致:Ixx, Ixy, Iyy, Ixz, Iyz, Izz)。
  \item \textbf{URDF}:\texttt{<inertial>} 中 \texttt{<origin xyz="cx cy cz"/>} 表示质心位置,\texttt{<inertia ixx=.../>} 表示 \textbf{质心处}惯性 $I_C$(与 10 维一致)。
  \item \textbf{Pinocchio 内部}:\texttt{Inertia} 对象存储的是连杆原点处的空间惯性;\texttt{inertia()} 返回 $I_{\mathrm{origin}}$($3\times 3$),\texttt{lever()} 返回 $\bm{c}$。
\end{itemize}

因此:CSV、URDF 文件、10 维参数中的“惯性”均为 $I_C$;Pinocchio 的 \texttt{model.inertias[jid].inertia()} 为 $I_{\mathrm{origin}}$。代码必须在读写 URDF 与 10 维时使用 $I_C$,在调用 Pinocchio 的 \texttt{FromDynamicParameters}/\texttt{toDynamicParameters} 时理解其输入输出分别为 $I_C$ 与 $I_{\mathrm{origin}}$ 的转换。

%==============================================================================
\section{代码中的数据流与转换位置}
%==============================================================================

\subsection{总览}

图~\ref{fig:flow} 概括了惯性在程序中的流向。下面对每条路径给出对应代码位置与公式。

\begin{figure}[h]
  \centering
  \fbox{\parbox{0.85\textwidth}{
    \textbf{URDF 文件}($I_C$)\\
    $\downarrow$ \texttt{buildModel} \\
    \textbf{Pinocchio 内部}($I_{\mathrm{origin}}$)\\
    $\downarrow$ \texttt{toDynamicParameters()},式\eqref{eq:origin_to_com} \\
    \textbf{theta\_urdf / CSV}(10 维,$I_C$ 在 $\pi[4{:}9]$)\\
    $\downarrow$ Ridge/QP,投影 \texttt{projectThetaToPhysical}(见正文)\\
    \textbf{theta\_estimated}(10 维,$I_C$)\\
    $\downarrow$ 写 URDF 时直接用 $\pi[4{:}9]$;写报告时用 $\pi[4{:}9]$ 或 $I_{\mathrm{origin}}-m S^T S$\\
    \textbf{输出 URDF / dynamics\_physical\_parameters\_identified.txt}($I_C$)
  }}
  \caption{惯性在程序中的表示与转换}
  \label{fig:flow}
\end{figure}

%------------------------------------------------------------------------------
\subsection{URDF $\to$ 10 维(theta\_urdf)}
%------------------------------------------------------------------------------

\textbf{位置}:约 568--584 行。

从 URDF 加载模型后,对每个关节 \texttt{jid} 调用 \texttt{pinocchio\_model.inertias[jid].toDynamicParameters()} 得到 10 维并写入 \texttt{theta\_urdf}。Pinocchio 源码(\texttt{inertia.hpp})中:
\begin{lstlisting}[language=C++,basicstyle=\ttfamily\small]
v.segment<6>(4) = (inertia() - AlphaSkewSquare(mass(), lever())).data();
\end{lstlisting}
即 $v_{4:9}$ 来自 $I_{\mathrm{origin}} - m\,S^\top S = I_C$。因此 \textbf{theta\_urdf 中每连杆后 6 个参数已是质心处惯量},无需在本仓库中再做平行轴换算。

%------------------------------------------------------------------------------
\subsection{10 维 $\to$ 写回 URDF(generateIdentifiedURDF)}
%------------------------------------------------------------------------------

\textbf{位置}:约 149--176 行。

程序将辨识得到的 \texttt{theta\_estimated} 写回 URDF。URDF 的 \texttt{<inertia>} 必须为质心处 $I_C$。若错误地使用 \texttt{inertia\_id.inertia()}(即 \texttt{FromDynamicParameters(pi\_id)} 后得到的 $I_{\mathrm{origin}}$),会与 URDF 语义不符。代码中已改为直接使用 10 维的后 6 个分量:

\begin{lstlisting}[language=C++,basicstyle=\ttfamily\small]
// URDF 约定:<inertia> 为质心处惯性 I_C;10 维中 pi_id[4..9] 即为 I_C
const double ixx = pi_id(4), ixy = pi_id(5), iyy = pi_id(6),
             ixz = pi_id(7), iyz = pi_id(8), izz = pi_id(9);
// 写入 <inertia ixx=... ixy=... ixz=... iyy=... iyz=... izz=... />
\end{lstlisting}

质量与质心仍由 \texttt{inertia\_id.mass()}、\texttt{inertia\_id.lever()} 得到(与 $\pi[0],\pi[1{:}3]/m$ 一致)。这样 \textbf{从 10 维到 URDF 的转换仅做取数与顺序映射,不涉及平行轴},且与 URDF 约定一致。

%------------------------------------------------------------------------------
\subsection{物理投影 projectThetaToPhysical 中的惯量}
%------------------------------------------------------------------------------

\textbf{位置}:约 273--337 行。

输入 \texttt{theta} 的每块 10 维中,$\pi[4{:}9]$ 为 $I_C$。流程为:

\begin{enumerate}[leftmargin=*]
  \item \texttt{FromDynamicParameters(pi)} 得到 \texttt{inv}。Pinocchio 内部存的是 $I_{\mathrm{origin}} = I_C + m\,S^\top S$,故 \texttt{inv.inertia()} 为 $I_{\mathrm{origin}}$。
  \item 代码对 \texttt{I = inv.inertia()}(即 $I_{\mathrm{origin}}$)做对称化、特征值下界、三角不等式等投影,得到 $I_{\mathrm{origin}}^{\mathrm{proj}}$。
  \item \texttt{inv\_proj(m, c, I\_origin\_proj)} 构造投影后的 Inertia,再 \texttt{pi\_new = inv\_proj.toDynamicParameters()}。Pinocchio 的 \texttt{toDynamicParameters()} 会计算 $I_C^{\mathrm{proj}} = I_{\mathrm{origin}}^{\mathrm{proj}} - m\,S^\top S$ 并填入返回向量的 $[4{:}9]$。
  \item \texttt{theta.segment(base, 10) = pi\_new} 将投影后的 \textbf{质心处惯量} 写回 10 维。
\end{enumerate}

因此:\textbf{投影在连杆原点惯量 $I_{\mathrm{origin}}$ 上进行,写回 theta 时通过 \texttt{toDynamicParameters()} 正确转回质心惯量 $I_C$},转换关系由 Pinocchio 保证。

%------------------------------------------------------------------------------
\subsection{运行前打印:URDF 原始 vs 转换后 10 维}
%------------------------------------------------------------------------------

\textbf{位置}:约 586--618 行。

为便于核对,程序对每个连杆打印:
\begin{itemize}[leftmargin=*]
  \item \textbf{I\_origin(连杆原点)}:\texttt{inv.inertia().matrix()},即 $I_{\mathrm{origin}}$;
  \item \textbf{I\_at\_COM(质心处)}:\texttt{inv.inertia() - AlphaSkewSquare(m\_urdf, c\_urdf)},即 $I_C$,与 \texttt{theta\_urdf} 的 $\pi[4{:}9]$ 一致;
  \item \textbf{转换后 10 维}:\texttt{theta\_urdf} 的对应 10 个数。
\end{itemize}

注释中已说明:\texttt{inertia()} 为连杆原点惯性,\texttt{toDynamicParameters()} 中 I6 为质心处惯性,二者差平行轴项,数值不同属正常。

%------------------------------------------------------------------------------
\subsection{辨识结果报告 dynamics\_physical\_parameters\_identified.txt}
%------------------------------------------------------------------------------

\textbf{位置}:约 918--955 行。

报告中“URDF 惯性”与“辨识 惯性”均统一为 \textbf{质心处},以便与 10 维及输出 URDF 一致:

\begin{itemize}[leftmargin=*]
  \item \textbf{URDF 惯性(质心处)}:由 \texttt{pinocchio\_model.inertias[jid].inertia() - AlphaSkewSquare(m\_urdf, c\_urdf)} 得到 $I_C$ 再打印为 $3\times 3$。
  \item \textbf{辨识 惯性(质心处)}:由 \texttt{pi\_id(4)...pi\_id(9)} 按 Symmetric3 顺序拼成 $3\times 3$ 再打印(不调用 \texttt{inertia\_id.inertia()})。
\end{itemize}

说明第 5 条注明:“以下惯性均为质心处惯性(与 URDF \texttt{<inertia>} 及 10 维参数一致)”。

%==============================================================================
\section{小结:何处为原点惯量、何处为质心惯量}
%==============================================================================

表~\ref{tab:summary} 汇总代码与数据中惯量的参考点。

\begin{table}[htbp]
  \centering
  \caption{惯性参考点在程序中的使用}
  \label{tab:summary}
  \begin{tabular}{llp{5.5cm}}
    \toprule
    位置 & 参考点 & 说明 \\
    \midrule
    10 维 $\theta$、CSV 后 6 列 & 质心 $I_C$ & 回归变量、Ridge/QP 解、投影后的 \texttt{theta\_estimated} \\
    URDF \texttt{<inertia>}(读/写) & 质心 $I_C$ & 约定;写出时用 \texttt{pi\_id(4:9)} \\
    \texttt{toDynamicParameters()} 返回值 $[4{:}9]$ & 质心 $I_C$ & Pinocchio 内部由 $I_{\mathrm{origin}}-m S^T S$ 得到 \\
    \texttt{model.inertias[jid].inertia()} & 原点 $I_{\mathrm{origin}}$ & Pinocchio 内部存储 \\
    \texttt{FromDynamicParameters(pi).inertia()} & 原点 $I_{\mathrm{origin}}$ & $I_C + m S^T S$ \\
    \texttt{projectThetaToPhysical} 中投影对象 & 原点 $I_{\mathrm{origin}}$ & 对 \texttt{inv.inertia()} 投影;写回时经 \texttt{toDynamicParameters()} 转成 $I_C$ \\
    运行前打印 “I\_origin” & 原点 $I_{\mathrm{origin}}$ & \texttt{inv.inertia().matrix()} \\
    运行前打印 “I\_at\_COM” / 报告中的惯性 & 质心 $I_C$ & 与 10 维、URDF 一致 \\
    \bottomrule
  \end{tabular}
\end{table}

\textbf{结论}:程序正确区分并处理了“连杆原点处转动惯量”与“质心处转动惯量”的转换。所有对外输出(CSV、URDF、物理参数报告)均为质心处 $I_C$;仅在 Pinocchio 内部及物理投影的中间步骤使用 $I_{\mathrm{origin}}$,并通过 \texttt{toDynamicParameters()} 或显式减去 \texttt{AlphaSkewSquare} 正确转回 $I_C$。

\end{document}
