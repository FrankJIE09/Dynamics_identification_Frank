\documentclass[12pt,a4paper]{article}
\usepackage[UTF8]{ctex}
\usepackage{amsmath}
\usepackage{amssymb}
\usepackage{geometry}
\usepackage{enumitem}
\usepackage{hyperref}
\usepackage{bm}
\usepackage{listings}
\usepackage{xcolor}
\usepackage{booktabs}

\geometry{margin=2.5cm}

\title{Y\_all 输入数据一致与回归矩阵数值差异说明}
\author{Dynamics\_identification\_Frank}
\date{\today}

\begin{document}

\maketitle

\begin{abstract}
在 step2 动力学参数辨识中,C++ 与 Python 使用同一份采集数据与 URDF,经对比脚本验证:用于构造回归矩阵的输入 $(q,\dot{q},\ddot{q})$ 逐样本、逐列完全一致(差为零),但回归矩阵 $Y_{\mathrm{all}}$ 的数值存在显著差异(最大绝对差约 $84.6$,相对 Frobenius 范数约 $1.42$),且 C++ 与 Python 下回归矩阵的秩分别为 45 与 7,辨识 RMSE 分别为约 $1.34\,\mathrm{Nm}$ 与 $8.67\,\mathrm{Nm}$。本文从 Pinocchio 官方文档与版本差异出发,详细解释为何“输入一致、输出不同”,并给出参考文献与建议。
\end{abstract}

\tableofcontents
\newpage

%==============================================================================
\section{现象概述}
%==============================================================================

\subsection{对比实验设置}

\begin{itemize}[leftmargin=*]
  \item \textbf{数据}:同一 CSV(\texttt{dynamics\_identification\_data.csv}),同一 URDF(\texttt{AR5-5\_07R-W4C4A2\_Manual\_fix.urdf})。
  \item \textbf{C++}:\texttt{step2\_dynamics\_parameter\_estimation},从 \texttt{build/} 运行,使用系统或 CMake 配置的 Pinocchio(如 2.x)。
  \item \textbf{Python}:\texttt{step2\_dynamics\_parameter\_estimation.py},使用 pip/conda 安装的 \texttt{pinocchio}(可能为 2.6 或 3.x)。
  \item \textbf{对比脚本}:\texttt{scripts/compare\_Y\_input\_and\_diff.py},读取 \texttt{Y\_input\_cpp.csv}、\texttt{Y\_input\_py.csv} 以及 \texttt{Y\_all\_cpp.bin}、\texttt{Y\_all\_py.bin},输出差值 CSV 与统计摘要。
\end{itemize}

\subsection{输入数据对比结果}

对比脚本对 C++ 与 Python 保存的“用于构造 $Y_{\mathrm{all}}$ 的输入”逐样本、逐列求差,结果为:

\begin{itemize}[leftmargin=*]
  \item 样本数一致:3001。
  \item 全表最大绝对差:$0$;全表平均绝对差:$0$。
  \item 各列($q_1\sim q_7,\, \dot{q}_1\sim \dot{q}_7,\, \ddot{q}_1\sim \ddot{q}_7$)的最大与平均绝对差均为 $0$。
\end{itemize}

因此可以确认:\textbf{两边使用的 $(q,\dot{q},\ddot{q})$ 完全一致},即“Y\_all 使用的数据”相同。

\subsection{回归矩阵与辨识结果对比}

\begin{itemize}[leftmargin=*]
  \item \textbf{回归矩阵 $Y_{\mathrm{all}}$}:形状一致 $(21007\times 70)$,但数值不同:
    \begin{itemize}
      \item 最大绝对差:约 $8.46\times 10^1$;
      \item 平均绝对差:约 $3.65$;
      \item Frobenius 范数差:约 $9.71\times 10^3$,相对 $Y_{\mathrm{cpp}}$ 的 Frobenius 范数约 $1.42$。
    \end{itemize}
  \item \textbf{秩}:C++ 约 45,Python 约 7(相同阈值 $10^{-6}\cdot \sigma_1$)。
  \item \textbf{辨识 RMSE}:C++ 约 $1.34\,\mathrm{Nm}$,Python 约 $8.67\,\mathrm{Nm}$。
\end{itemize}

结论:在输入 $(q,\dot{q},\ddot{q})$ 一致的前提下,\textbf{回归矩阵 $Y$ 的数值差异来自 C++ 与 Python 所调用的 Pinocchio 实现不同}(见下节)。

%==============================================================================
\section{Pinocchio 官方文档中的关节力矩回归矩阵}
%==============================================================================

\subsection{函数定义与数学关系}

Pinocchio 官方文档(如 ROS Humble 下 pinocchio 3.4.0 文档\cite{ros-pinocchio-regressor})对 \texttt{computeJointTorqueRegressor} 的说明可概括为:

\begin{quote}
\textbf{computeJointTorqueRegressor(model, data, q, v, a)}

Computes the joint torque regressor that links the joint torque to the dynamic parameters of each link according to the current robot motion.

The result is stored in \texttt{data.jointTorqueRegressor} and it corresponds to a matrix $Y$ such that
\begin{equation}
  \tau = Y(q,\dot{q},\ddot{q})\,\pi,
  \label{eq:tauYpi}
\end{equation}
where $\pi = (\pi_1^\top\;\ldots\;\pi_n^\top)^\top$ and
\begin{equation}
  \pi_i = \texttt{model.inertias[i].toDynamicParameters()}.
  \label{eq:pi_i}
\end{equation}
\end{quote}

即:给定当前运动 $(q,v,a)$,$Y$ 将各连杆的动力学参数 $\pi_i$(质量、质心、惯量等 10 维)与关节力矩 $\tau$ 线性联系起来。文档还注明:该函数在计算过程中会向 \texttt{data.bodyRegressor} 写入临时信息,原有内容会被覆盖。

\subsection{参数与接口}

\begin{itemize}[leftmargin=*]
  \item \textbf{参数}:\texttt{model}(刚体系统模型),\texttt{data}(数据缓存),\texttt{q}(关节位置,维数 \texttt{model.nq}),\texttt{v}(关节速度,\texttt{model.nv}),\texttt{a}(关节加速度,\texttt{model.nv})。
  \item \textbf{返回}:系统关节力矩回归矩阵(即 $Y$),存储在 \texttt{data.jointTorqueRegressor}。
  \item \textbf{动力学参数}:每个连杆 $i$ 的 $\pi_i$ 来自 \texttt{model.inertias[i].toDynamicParameters()},通常为 10 维(质量 + 质心相关 3 维 + 惯量矩阵 6 维),具体顺序与参考系以 Pinocchio 各版本实现为准。
\end{itemize}

\subsection{为何“同一公式”会得到不同 $Y$?}

式 \eqref{eq:tauYpi} 与 \eqref{eq:pi_i} 在文档层面是统一的,但 \textbf{具体实现}(即如何由 $(q,v,a)$ 和运动学/动力学中间量算出 $Y$ 的每一元素)取决于:

\begin{enumerate}[leftmargin=*]
  \item \textbf{Pinocchio 版本}:Pinocchio 2 与 3 在命名空间、部分算法与 API 上存在差异。官方“Porting from Pinocchio 2 to 3”\cite{pinocchio-porting23} 指出:例如 \texttt{pinocchio::regressor::computeStaticRegressor} 改为 \texttt{pinocchio::computeStaticRegressor},以及若干与运动学、雅可比、能量等相关的函数更名或签名变更。关节力矩回归矩阵虽未在移植文档中单独列出,但若内部依赖的运动学/动力学接口或数据布局在不同版本间有变化,则同一 $(q,v,a)$ 下算出的 $Y$ 可能不同。
  \item \textbf{编译选项与依赖}:C++ 侧通常链接系统或 CMake 找到的 Pinocchio(可能带 \texttt{PINOCCHIO\_WITHOUT\_EIGEN\_TENSOR\_SUPPORT} 等宏),Python 侧使用预编译的 wheel(如 \texttt{pip install pinocchio}),两边的 Pinocchio 版本号、补丁与编译配置可能不一致,导致实际执行的代码路径不同。
  \item \textbf{数值实现细节}:例如参考系(连杆坐标系/世界系)、关节顺序、浮点运算顺序、是否使用 SIMD 等,都可能影响最终矩阵元素。若某一版本对 regressor 的实现有修正或优化,也会造成与另一版本的数值差异。
\end{enumerate}

因此,\textbf{“输入 $(q,\dot{q},\ddot{q})$ 一致”只能保证两边用的是同一组运动状态;回归矩阵 $Y$ 由各自 Pinocchio 库的内部实现计算,版本/构建不同即可导致 $Y$ 不同},进而导致秩、辨识解和 RMSE 的差异。

%==============================================================================
\section{版本与 API 参考(Pinocchio 2/3)}
%==============================================================================

根据 Pinocchio 官方移植说明\cite{pinocchio-porting23}:

\begin{itemize}[leftmargin=*]
  \item \textbf{C++}:Pinocchio 3 中部分算法从子命名空间移入主命名空间,部分头文件与函数签名变更;若 C++ 工程使用 Pinocchio 2 而 Python 使用 Pinocchio 3(或反之),则两边调用的 regressor 实现可能并非同一套代码。
  \item \textbf{Python}:Python 绑定中也有大量更名(如 \texttt{frameJacobian} $\to$ \texttt{computeFrameJacobian} 等)。若 Python 端通过 \texttt{data.joint\_torque\_regressor} 或 \texttt{data.jointTorqueRegressor} 访问结果,需与当前安装的 Pinocchio 版本文档一致,否则可能访问到不同实现或不同数据布局。
\end{itemize}

建议在对比 C++ 与 Python 时,明确记录两边的 Pinocchio 版本(例如 C++:\texttt{pkg-config --modversion pinocchio} 或 CMake 输出;Python:\texttt{pip show pinocchio}),并在文档中注明“回归矩阵差异来自 Pinocchio 版本/实现差异,而非输入数据不同”。

%==============================================================================
\section{结论与建议}
%==============================================================================

\subsection{结论}

\begin{enumerate}[leftmargin=*]
  \item 通过 \texttt{Y\_input\_cpp.csv} 与 \texttt{Y\_input\_py.csv} 的逐样本、逐列求差,可确认 C++ 与 Python 构造 $Y_{\mathrm{all}}$ 时使用的 $(q,\dot{q},\ddot{q})$ \textbf{完全一致}。
  \item $Y_{\mathrm{all}}$ 的数值差异、秩差异以及辨识 RMSE 差异,来源于 \textbf{C++ 与 Python 所依赖的 Pinocchio 库版本或构建不同},导致 \texttt{computeJointTorqueRegressor} 的实际实现不同,而非输入数据错误。
  \item Pinocchio 官方文档明确给出 $\tau=Y(q,\dot{q},\ddot{q})\pi$ 及 $\pi_i=\texttt{model.inertias[i].toDynamicParameters()}$;不同版本/实现在该公式下的具体数值实现可能不同。
\end{enumerate}

\subsection{建议}

\begin{itemize}[leftmargin=*]
  \item \textbf{复现与对比}:若需 C++ 与 Python 得到尽量一致的 $Y$ 与辨识结果,应尽量保证两边使用相同版本的 Pinocchio(例如都使用 2.6.x 或都使用 3.x),并统一 URDF、数据与阈值。
  \item \textbf{工程选用}:若以辨识精度或与实验对比为准,建议以 C++ 结果(本项目中秩约 45、RMSE 约 $1.34\,\mathrm{Nm}$)为参考,或统一采用某一 Pinocchio 版本后再做对比。
  \item \textbf{文档与引用}:在报告或论文中可引用 Pinocchio 官方文档\cite{ros-pinocchio-regressor}\cite{pinocchio-porting23},并说明“回归矩阵由 Pinocchio 的 \texttt{computeJointTorqueRegressor} 计算;C++/Python 数值差异来自库版本与实现差异”。
\end{itemize}

%==============================================================================
\section*{参考文献}
%==============================================================================

\begin{thebibliography}{99}
  \bibitem{ros-pinocchio-regressor} Pinocchio 3.4.0 (ROS Humble), \textit{Template Function pinocchio::computeJointTorqueRegressor}. \\
  \url{https://docs.ros.org/en/humble/p/pinocchio/generated/function_namespacepinocchio_1af97c3d2d695ef4636bf010c2ff6031e8.html}
  \bibitem{pinocchio-porting23} Pinocchio, \textit{Porting from Pinocchio 2 to 3}. \\
  \url{https://gepettoweb.laas.fr/doc/stack-of-tasks/pinocchio/master/doxygen-html/md_doc__porting.html}
  \bibitem{pinocchio-dynamic} Pinocchio, \textit{Inverse dynamics / Dynamic documentation}. \\
  \url{https://docs.ros.org/en/jazzy/p/pinocchio/doc/a-features/g-dynamic.html}
  \bibitem{pinocchio-regressor-cpp} Pinocchio, \textit{regressor.cpp File Reference}. \\
  \url{https://docs.ros.org/en/api/pinocchio/html/src_2algorithm_2regressor_8cpp.html}
\end{thebibliography}

\end{document}
