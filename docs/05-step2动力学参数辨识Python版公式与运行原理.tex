\documentclass[12pt,a4paper]{article}
\usepackage[UTF8]{ctex}
\usepackage{amsmath}
\usepackage{amssymb}
\usepackage{geometry}
\usepackage{enumitem}
\usepackage{hyperref}
\usepackage{bm}
\usepackage{booktabs}

\geometry{margin=2.5cm}

\title{step2 动力学参数辨识(Python 版)\\运行原理与计算公式}
\author{Dynamics\_identification\_Frank}
\date{\today}

\begin{document}

\maketitle

\begin{abstract}
本文针对 \texttt{scripts/step2\_dynamics\_parameter\_estimation.py} 的运行原理进行说明,重点给出各步骤对应的数学公式:动力学线性回归形式、回归矩阵与数据堆叠、最小二乘(SVD)、Ridge 回归、带质量约束的 QP 求解、物理可行域投影,以及误差与验证指标。
\end{abstract}

\tableofcontents
\newpage

%==============================================================================
\section{动力学线性回归模型}
%==============================================================================

\subsection{关节力矩与动力学参数}

机器人刚体动力学可写成关于动力学参数的线性形式:
\begin{equation}
  \bm{\tau} = Y(\bm{q},\dot{\bm{q}},\ddot{\bm{q}})\,\bm{\pi},
  \label{eq:tau_Y_pi}
\end{equation}
其中
\begin{itemize}[leftmargin=*]
  \item $\bm{\tau}\in\mathbb{R}^{n_q}$:关节力矩($n_q=7$);
  \item $\bm{q},\dot{\bm{q}},\ddot{\bm{q}}$:关节位置、速度、加速度;
  \item $Y\in\mathbb{R}^{n_q\times n_{\mathrm{params}}}$:关节力矩回归矩阵,由 Pinocchio 的 \texttt{computeJointTorqueRegressor} 计算;
  \item $\bm{\pi}\in\mathbb{R}^{n_{\mathrm{params}}}$:全机动力学参数向量(本程序中 $n_{\mathrm{params}}=70$,即 7 个连杆 $\times$ 每连杆 10 维)。
\end{itemize}

\subsection{单连杆 10 维参数}

每个连杆 $j$ 对应 10 个动力学参数(质心处惯量约定,与文档 03 一致):
\begin{equation}
  \bm{\pi}_j = \bigl[\, m_j,\; m_j c_x,\; m_j c_y,\; m_j c_z,\; I_{xx},\; I_{xy},\; I_{yy},\; I_{xz},\; I_{yz},\; I_{zz} \,\bigr]^\top \in \mathbb{R}^{10}.
  \label{eq:pi_10}
\end{equation}
其中 $m_j$ 为质量,$(c_x,c_y,c_z)$ 为质心在连杆坐标系下的坐标,后 6 项为质心处惯性张量 $I_C$ 的独立分量(\texttt{ixx, ixy, iyy, ixz, iyz, izz})。全机参数为
\[
\bm{\pi} = [\bm{\pi}_1^\top,\, \bm{\pi}_2^\top,\, \ldots,\, \bm{\pi}_{n_{\mathrm{links}}}^\top]^\top.
\]

%==============================================================================
\section{数据与回归矩阵的堆叠}
%==============================================================================

\subsection{单样本数据}

CSV 每行一个采样点,列顺序为:
\[
\texttt{timestamp},\; q_1,\ldots,q_7,\; \dot{q}_1,\ldots,\dot{q}_7,\; \ddot{q}_1,\ldots,\ddot{q}_7,\; \tau_1,\ldots,\tau_7.
\]
对第 $i$ 个样本,给定 $(\bm{q}^{(i)},\dot{\bm{q}}^{(i)},\ddot{\bm{q}}^{(i)})$,调用 Pinocchio 得到该时刻的回归矩阵 $Y^{(i)}\in\mathbb{R}^{7\times 70}$ 与测量力矩 $\bm{\tau}^{(i)}\in\mathbb{R}^7$,满足
\[
\bm{\tau}^{(i)} = Y^{(i)}\,\bm{\pi}.
\]

\subsection{堆叠形式}

设共有 $N$ 个样本。将各样本的 $Y^{(i)}$ 按行堆叠为 $Y_{\mathrm{all}}$,将 $\bm{\tau}^{(i)}$ 按顺序拼接为 $\bm{\tau}_{\mathrm{all}}$:
\begin{equation}
  Y_{\mathrm{all}} =
  \begin{bmatrix} Y^{(1)} \\ Y^{(2)} \\ \vdots \\ Y^{(N)} \end{bmatrix}
  \in \mathbb{R}^{(7N)\times 70},\qquad
  \bm{\tau}_{\mathrm{all}} =
  \begin{bmatrix} \bm{\tau}^{(1)} \\ \bm{\tau}^{(2)} \\ \vdots \\ \bm{\tau}^{(N)} \end{bmatrix}
  \in \mathbb{R}^{7N}.
  \label{eq:Y_tau_stack}
\end{equation}
则全数据下的线性方程为
\begin{equation}
  Y_{\mathrm{all}}\,\bm{\theta} = \bm{\tau}_{\mathrm{all}},
  \label{eq:Y_theta_tau}
\end{equation}
其中 $\bm{\theta}$ 为待辨识的 70 维参数向量(与 $\bm{\pi}$ 同义)。

%==============================================================================
\section{最小二乘解(SVD 伪逆)}
%==============================================================================

\subsection{秩与奇异值}

对 $Y_{\mathrm{all}}$ 做奇异值分解:
\begin{equation}
  Y_{\mathrm{all}} = U\,S\,V^\top,\quad
  S = \mathrm{diag}(s_1,s_2,\ldots,s_{70}),\quad s_1\ge s_2\ge \cdots \ge 0.
  \label{eq:svd}
\end{equation}
秩的估计:给定阈值 $\epsilon = 10^{-6}\cdot s_1$,
\[
\mathrm{rank}(Y_{\mathrm{all}}) = \#\{i : s_i > \epsilon\}.
\]
若秩 $<70$,则回归矩阵不满秩,部分参数不可单独辨识。

\subsection{伪逆解}

最小二乘解(无正则、无约束)为
\begin{equation}
  \bm{\theta}_{\mathrm{ls}} = Y_{\mathrm{all}}^+ \,\bm{\tau}_{\mathrm{all}}
  = V\,\bigl(S^+ U^\top \bm{\tau}_{\mathrm{all}}\bigr),
  \label{eq:theta_ls}
\end{equation}
其中 $S^+$ 为 $S$ 的伪逆:对角元为 $1/s_i$(若 $s_i>10^{-12}$),否则取 $1/10^{-12}$ 避免除零。即
\[
\theta_{\mathrm{ls}} = V^\top \,\bigl( (1/s_i)_{i} \odot (U^\top \bm{\tau}_{\mathrm{all}}) \bigr).
\]
代码中对应:\texttt{theta\_ls = (Vt.T * (1.0 / np.where(s > 1e-12, s, 1e-12))) @ (U.T @ tau\_all)}。

%==============================================================================
\section{Ridge 回归(URDF 先验)}
%==============================================================================

\subsection{正则化与先验}

引入 URDF 先验 $\bm{\theta}_{\mathrm{urdf}}$ 与正则系数 $\lambda$,Ridge 问题为
\begin{equation}
  \min_{\bm{\theta}} \;
  \bigl\| Y_{\mathrm{all}}\bm{\theta} - \bm{\tau}_{\mathrm{all}} \bigr\|^2
  + \lambda\,\bigl\| \bm{\theta} - \bm{\theta}_{\mathrm{urdf}} \bigr\|^2.
  \label{eq:ridge_obj}
\end{equation}

\subsection{正则系数}

程序中 $\lambda$ 由相对系数 \texttt{lambda\_rel} 与数据尺度确定:
\begin{equation}
  \lambda = \lambda_{\mathrm{rel}} \cdot \frac{\mathrm{tr}(Y_{\mathrm{all}}^\top Y_{\mathrm{all}})}{n_{\mathrm{params}}}.
  \label{eq:lambda}
\end{equation}
默认 $\lambda_{\mathrm{rel}}=10^{-3}$。

\subsection{正规方程与解}

式 \eqref{eq:ridge_obj} 对 $\bm{\theta}$ 求导并令其为零,得
\begin{equation}
  \bigl( Y_{\mathrm{all}}^\top Y_{\mathrm{all}} + \lambda I \bigr)\,\bm{\theta}
  = Y_{\mathrm{all}}^\top \bm{\tau}_{\mathrm{all}} + \lambda\,\bm{\theta}_{\mathrm{urdf}}.
  \label{eq:ridge_normal}
\end{equation}
记
\[
A = Y_{\mathrm{all}}^\top Y_{\mathrm{all}} + \lambda I,\qquad
\bm{b} = Y_{\mathrm{all}}^\top \bm{\tau}_{\mathrm{all}} + \lambda\,\bm{\theta}_{\mathrm{urdf}},
\]
则
\begin{equation}
  \bm{\theta}_{\mathrm{ridge}} = A^{-1}\bm{b}.
  \label{eq:theta_ridge}
\end{equation}
代码中通过 \texttt{np.linalg.solve(A, b)} 求解。

%==============================================================================
\section{带质量约束的 QP 求解(OSQP)}
%==============================================================================

\subsection{二次型与线性项}

将 Ridge 目标改写为标准 QP 形式:
\begin{equation}
  \min_{\bm{\theta}} \;\; \frac{1}{2}\,\bm{\theta}^\top H\,\bm{\theta} + \bm{g}^\top\bm{\theta},
  \label{eq:qp_obj}
\end{equation}
其中
\begin{align}
  H &= 2\,Y_{\mathrm{all}}^\top Y_{\mathrm{all}} + 2\lambda\,I
  \quad \text{(对角元为 } (Y^\top Y)_{ii} + 2\lambda \text{)}, \\
  \bm{g} &= -2\,\bigl( Y_{\mathrm{all}}^\top \bm{\tau}_{\mathrm{all}} + \lambda\,\bm{\theta}_{\mathrm{urdf}} \bigr).
  \label{eq:H_g}
\end{align}
代码实现:\texttt{H = 2.0 * (Y\_all.T @ Y\_all)},再 \texttt{H.flat[::n+1] += 2.0*lam};\texttt{g = -2.0 * (Y\_all.T @ tau\_all + lam * theta\_urdf)}。

\subsection{质量约束}

每连杆质量对应 $\bm{\theta}$ 中下标 $10j$ 的分量($j=0,\ldots,n_{\mathrm{links}}-1$),约束为
\begin{equation}
  \theta_{10j} \ge m_{\min},\qquad j=0,1,\ldots,n_{\mathrm{links}}-1.
  \label{eq:mass_constraint}
\end{equation}
写成 $A\bm{\theta}\le \bm{u}$ 形式:定义 $A\in\mathbb{R}^{n_{\mathrm{links}}\times n_{\mathrm{params}}}$,第 $j$ 行仅在列 $10j$ 处为 $-1$,其余为 0,则
\[
-\theta_{10j} \le -m_{\min} \;\Leftrightarrow\; (A\bm{\theta})_j \le -m_{\min}.
\]
OSQP 中设 $l_j=-\infty$,$u_j=-m_{\min}$,即得式 \eqref{eq:mass_constraint}。

\subsection{求解}

使用 OSQP 求解式 \eqref{eq:qp_obj} 与质量约束,得到 $\bm{\theta}_{\mathrm{qp}}$。若求解失败则回退到 Ridge 解 \eqref{eq:theta_ridge}。

%==============================================================================
\section{物理可行域投影}
%==============================================================================

对求得的 $\bm{\theta}$(或 $\bm{\theta}_{\mathrm{qp}}$)按连杆进行物理投影,保证质量与惯量可行。

\subsection{单连杆 10 维到 Inertia}

对连杆 $j$,取 $\bm{\pi}_j = \bm{\theta}[10j:10j+10]$,由 Pinocchio 转为惯性对象:
\[
(m,\, \bm{c},\, I) \leftarrow \texttt{Inertia.FromDynamicParameters}(\bm{\pi}_j).
\]
其中 $I$ 为连杆原点处惯性张量 $I_{\mathrm{origin}}$(对称 $3\times 3$)。

\subsection{质量下界}

\begin{equation}
  m \leftarrow \max(m,\, m_{\min}).
  \label{eq:m_min}
\end{equation}

\subsection{惯量对称与正定性}

\begin{enumerate}[leftmargin=*]
  \item 对称化:$I \leftarrow (I+I^\top)/2$。
  \item 特征值分解:$I = Q\,\Lambda\,Q^\top$,$\Lambda=\mathrm{diag}(\lambda_1,\lambda_2,\lambda_3)$,$\lambda_1\le\lambda_2\le\lambda_3$。
  \item 特征值下界:$\lambda_i \leftarrow \max(\lambda_i,\, I_{\epsilon})$。
  \item 三角不等式(转动惯量物理可行):满足 $\lambda_3 \le \lambda_1+\lambda_2$,否则取
  \begin{equation}
    \lambda_3 \leftarrow \min\bigl(\lambda_3,\, \lambda_1 + \lambda_2 - I_{\epsilon}\bigr),\qquad
    \lambda_3 \leftarrow \max(\lambda_3,\, I_{\epsilon}).
  \label{eq:triangle}
  \end{equation}
  \item 重构:$I \leftarrow Q\,\Lambda\,Q^\top$。
\end{enumerate}

\subsection{trace 过小回退}

若 $\mathrm{tr}(I) < I_{\mathrm{trace\_min}}$,且提供了 URDF 先验 $\bm{\theta}_{\mathrm{urdf}}$,则该连杆的 10 维参数用先验替代:
\[
\bm{\theta}[10j:10j+10] \leftarrow \bm{\theta}_{\mathrm{urdf}}[10j:10j+10].
\]
否则用投影后的 $(m,\bm{c},I)$ 再转回 10 维写回 $\bm{\theta}$(\texttt{inv\_proj.toDynamicParameters()})。

%==============================================================================
\section{重力设置}
%==============================================================================

与 C++ 一致,基座到世界的旋转取为 $R_{\mathrm{base}} = R_z(-\pi/2)\,R_y(\pi/2)$:
\begin{equation}
  R_y = \begin{bmatrix} 0&0&1 \\ 0&1&0 \\ -1&0&0 \end{bmatrix},\quad
  R_z = \begin{bmatrix} 0&-1&0 \\ 1&0&0 \\ 0&0&1 \end{bmatrix},\quad
  \bm{g}_{\mathrm{world}} = \begin{bmatrix} 0\\0\\-9.81 \end{bmatrix},\quad
  \bm{g}_{\mathrm{base}} = R_{\mathrm{base}}^\top \bm{g}_{\mathrm{world}}.
  \label{eq:gravity}
\end{equation}
模型中的重力设为 $\bm{g}_{\mathrm{base}}$,供 Pinocchio 计算回归矩阵与 RNEA。

%==============================================================================
\section{误差与验证指标}
%==============================================================================

\subsection{训练误差}

预测力矩与测量力矩的误差向量为
\[
\bm{e} = \bm{\tau}_{\mathrm{all}} - Y_{\mathrm{all}}\,\bm{\theta}_{\mathrm{est}}.
\]
定义
\begin{align}
  \mathrm{RMSE} &= \sqrt{\frac{1}{n_e}\sum_{k=1}^{n_e} e_k^2}, &
  n_e &= 7N; \\
  \mathrm{MaxError} &= \max_k |e_k|; \\
  \mathrm{MeanAbsError} &= \frac{1}{n_e}\sum_{k=1}^{n_e} |e_k|.
\end{align}

\subsection{验证集}

取后 20\% 样本作为验证集,用同一 $\bm{\theta}_{\mathrm{est}}$ 计算 $Y_{\mathrm{val}}\,\bm{\theta}_{\mathrm{est}}$,与测量 $\bm{\tau}_{\mathrm{val}}$ 比较,得到验证集 RMSE。并可用原 URDF 与辨识后的 URDF 分别做 RNEA 预测,与测量力矩比较 RMSE 与最大绝对误差。

%==============================================================================
\section{输出与 URDF 生成}
%==============================================================================

\subsection{参数与物理量}

\begin{itemize}[leftmargin=*]
  \item \texttt{dynamics\_parameters\_csv}:最终 $\bm{\theta}_{\mathrm{est}}$(含 QP/投影后)。
  \item \texttt{dynamics\_parameters\_ls\_csv}:Ridge 解 $\bm{\theta}_{\mathrm{ridge}}$(未做 QP 与投影)。
  \item \texttt{dynamics\_parameters\_urdf\_csv}:URDF 先验 $\bm{\theta}_{\mathrm{urdf}}$。
\end{itemize}
质心与惯量由 $\bm{\theta}_{\mathrm{est}}$ 按式 \eqref{eq:pi_10} 解析:$m=\pi_0$,$\bm{c}=(\pi_1,\pi_2,\pi_3)^\top/m$,质心处惯量 $I_C$ 的 6 个分量为 $\pi_4,\ldots,\pi_9$。

\subsection{两种 URDF}

\begin{enumerate}[leftmargin=*]
  \item \textbf{质心处惯量}:\texttt{<inertia>} 为 $I_C$,\texttt{<origin xyz>} 为质心 $\bm{c}$。
  \item \textbf{关节原点惯量}:\texttt{<inertia>} 为 $I_{\mathrm{origin}}$(由 Pinocchio 的 \texttt{inv\_id.inertia()}),\texttt{<origin xyz>} 仍为质心。与文档 03 中平行轴定理一致。
\end{enumerate}

%==============================================================================
\section{公式汇总表}
%==============================================================================

\begin{table}[h]
\centering
\caption{主要公式与代码对应}
\begin{tabular}{ll}
\toprule
\textbf{内容} & \textbf{公式/说明} \\
\midrule
动力学回归 & $\bm{\tau}=Y(\bm{q},\dot{\bm{q}},\ddot{\bm{q}})\bm{\pi}$ \\
10 维参数 & $\bm{\pi}_j=[m,mc_x,mc_y,mc_z,I_{xx},I_{xy},I_{yy},I_{xz},I_{yz},I_{zz}]^\top$ \\
堆叠方程 & $Y_{\mathrm{all}}\bm{\theta}=\bm{\tau}_{\mathrm{all}}$ \\
SVD 秩 & $\mathrm{rank}=\#\{i:s_i>10^{-6}s_1\}$ \\
伪逆解 & $\bm{\theta}_{\mathrm{ls}}=V S^+ U^\top \bm{\tau}_{\mathrm{all}}$ \\
$\lambda$ & $\lambda=\lambda_{\mathrm{rel}}\,\mathrm{tr}(Y^\top Y)/n_{\mathrm{params}}$ \\
Ridge 解 & $(Y^\top Y+\lambda I)\bm{\theta}=Y^\top\bm{\tau}+\lambda\bm{\theta}_{\mathrm{urdf}}$ \\
QP:$H$ & $H=2Y^\top Y+2\lambda I$ \\
QP:$\bm{g}$ & $\bm{g}=-2(Y^\top\bm{\tau}+\lambda\bm{\theta}_{\mathrm{urdf}})$ \\
质量约束 & $\theta_{10j}\ge m_{\min}$ \\
质量投影 & $m\leftarrow\max(m,m_{\min})$ \\
惯量特征值 & $\lambda_i\leftarrow\max(\lambda_i,I_{\epsilon})$;三角不等式 $\lambda_3\le\lambda_1+\lambda_2$ \\
RMSE & $\sqrt{\mathrm{mean}(\bm{e}^2)}$,$\bm{e}=\bm{\tau}_{\mathrm{all}}-Y_{\mathrm{all}}\bm{\theta}$ \\
\bottomrule
\end{tabular}
\label{tab:formulas}
\end{table}

\end{document}
