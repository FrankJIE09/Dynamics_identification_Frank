\documentclass[12pt,a4paper]{article}
\usepackage[UTF8]{ctex}
\usepackage{amsmath}
\usepackage{amssymb}
\usepackage{geometry}
\usepackage{enumitem}
\usepackage{hyperref}
\usepackage{bm}

\geometry{margin=2.5cm}

\title{URDF 与 Pinocchio 惯量均为绕质心约定}
\author{Dynamics\_identification\_Frank}
\date{\today}

\begin{document}

\maketitle

\begin{abstract}
\textbf{核心结论:}URDF 中 \texttt{<inertial>} 的 \texttt{<inertia>} 与 Pinocchio 中 \texttt{Inertia} 的转动惯量矩阵,\textbf{均为绕质心} $I_C$。辨识程序写回 URDF 时必须使用绕质心惯量,才能与 Pinocchio 读回后的 \texttt{toDynamicParameters()} 一致,保证 $\theta$ 写/读零差异。
\end{abstract}

\section{约定说明}

\subsection{URDF 约定}

在 URDF 中,\texttt{<inertial>} 块为:
\begin{itemize}[leftmargin=*]
  \item \texttt{<origin xyz="cx cy cz"/>}:质心在连杆坐标系下的位置;
  \item \texttt{<inertia ixx=... ixy=... />}:\textbf{绕该惯性系原点(即质心)}的 $3\times 3$ 转动惯量,即 $I_C$。
\end{itemize}
因此 URDF 文件中的惯量语义是\textbf{绕质心} $I_C$。

\subsection{Pinocchio 约定}

Pinocchio 的 \texttt{Inertia} 类由 \texttt{FromDynamicParameters(}$\pi$\texttt{)} 构造,其中 $\pi$ 为 10 维动力学参数 $(m,\,m c_x,\,m c_y,\,m c_z,\,I_{xx},\,I_{xy},\,I_{xz},\,I_{yy},\,I_{yz},\,I_{zz})$。该类内部保存质心位置 \texttt{lever} 与转动惯量矩阵。\textbf{该转动惯量矩阵为绕质心} $I_C$,与 URDF 约定一致。

\subsection{写回 URDF 时必须使用绕质心惯量}

辨识得到的 $\theta$(或 $\pi$)经 \texttt{Inertia::FromDynamicParameters} 得到 \texttt{inv\_id} 后:
\begin{itemize}[leftmargin=*]
  \item 写入 URDF 的 \texttt{<inertia>} 应使用 \textbf{\texttt{inv\_id.inertia()} 的 3$\times$3 矩阵}(即 $I_C$),与 \texttt{<origin xyz>} 质心位置一致;
  \item 若误将绕连杆原点的 $I_{\mathrm{origin}}$ 写入 URDF,解析器会按“绕质心”解释并再做平行轴转换,导致读回 $\theta$ 与内存 $\theta$ 出现约 2 倍等错误,写/读不一致。
\end{itemize}

\section{代码中的体现}

\begin{itemize}[leftmargin=*]
  \item \textbf{C++}(\texttt{step2\_dynamics\_parameter\_estimation.cpp}):使用 \texttt{pi\_id(4)\ldots pi\_id(9)} 写入 URDF,与 Pinocchio 10 维参数中 $I_C$ 部分一致;注释中明确“\texttt{<inertia>} 为质心处惯性 $I_C$”。
  \item \textbf{Python 含摩擦}(\texttt{step2\_dynamics\_parameter\_estimation\_friction.py}):使用 \texttt{inv\_id.inertia} 的 3$\times$3 矩阵写入,与 $\theta$ 的 $I$ 部分一致,均为绕质心。
  \item \textbf{Python 无摩擦}(\texttt{step2\_dynamics\_parameter\_estimation.py}):同样使用与 10 维参数中 $I_C$ 对应的惯量写入 URDF。
\end{itemize}

\section{小结}

URDF 与 Pinocchio 的惯量\textbf{都是绕质心} $I_C$。辨识程序写 URDF 时统一使用质心处惯量(来自 \texttt{Inertia} 的 \texttt{inertia()} 或 $\theta$ 的对应分量),即可保证保存后再读回的 $\theta$ 与内存一致。

\end{document}
