\documentclass[12pt,a4paper]{article}
\usepackage[UTF8]{ctex}
\usepackage{amsmath}
\usepackage{amssymb}
\usepackage{amsthm}
\usepackage{geometry}
\usepackage{graphicx}
\usepackage{hyperref}
\usepackage{listings}
\usepackage{xcolor}

\geometry{margin=2.5cm}

\title{Pinocchio 动力学计算原理}
\author{xCore SDK 混合力位控制示例}
\date{\today}

\begin{document}

\maketitle

\section{概述}

Pinocchio 是一个高效的刚体动力学计算库,用于计算机器人的正向动力学、逆向动力学、雅可比矩阵等。本文档介绍 Pinocchio 如何计算机器人的动力学项,以及这些计算在混合力位控制中的应用。

\section{机器人动力学方程}

机器人的完整动力学方程可以表示为:

\begin{equation}
\boldsymbol{\tau} = \mathbf{M}(\boldsymbol{q}) \ddot{\boldsymbol{q}} + \mathbf{C}(\boldsymbol{q}, \dot{\boldsymbol{q}}) \dot{\boldsymbol{q}} + \boldsymbol{g}(\boldsymbol{q}) + \boldsymbol{\tau}_f(\boldsymbol{q}, \dot{\boldsymbol{q}})
\label{eq:dynamics}
\end{equation}

其中:
\begin{itemize}
    \item $\boldsymbol{\tau} \in \mathbb{R}^n$:关节力矩向量($n$ 为自由度)
    \item $\boldsymbol{q} \in \mathbb{R}^n$:关节位置向量
    \item $\dot{\boldsymbol{q}} \in \mathbb{R}^n$:关节速度向量
    \item $\ddot{\boldsymbol{q}} \in \mathbb{R}^n$:关节加速度向量
    \item $\mathbf{M}(\boldsymbol{q}) \in \mathbb{R}^{n \times n}$:质量矩阵(惯性矩阵)
    \item $\mathbf{C}(\boldsymbol{q}, \dot{\boldsymbol{q}}) \in \mathbb{R}^{n \times n}$:科氏力矩阵
    \item $\boldsymbol{g}(\boldsymbol{q}) \in \mathbb{R}^n$:重力项
    \item $\boldsymbol{\tau}_f(\boldsymbol{q}, \dot{\boldsymbol{q}}) \in \mathbb{R}^n$:摩擦力项
\end{itemize}

\section{Pinocchio 的动力学计算}

\subsection{computeAllTerms 函数}

Pinocchio 提供了 \texttt{computeAllTerms()} 函数,可以一次性计算所有动力学项:

\begin{lstlisting}[language=C++, basicstyle=\ttfamily\small]
pinocchio::computeAllTerms(model, data, q, dq);
\end{lstlisting}

该函数计算并存储以下内容:
\begin{itemize}
    \item \texttt{data.M}:质量矩阵 $\mathbf{M}(\boldsymbol{q})$
    \item \texttt{data.C}:科氏力矩阵 $\mathbf{C}(\boldsymbol{q}, \dot{\boldsymbol{q}})$
    \item \texttt{data.g}:重力项 $\boldsymbol{g}(\boldsymbol{q})$
    \item \texttt{data.nle}:非线性效应项(科氏力+重力)
\end{itemize}

\subsection{质量矩阵(惯性矩阵)}

质量矩阵 $\mathbf{M}(\boldsymbol{q})$ 表示机器人在当前配置下的惯性特性。它是对称正定矩阵,其元素 $M_{ij}$ 表示关节 $j$ 的加速度对关节 $i$ 的力矩的影响。

在 Pinocchio 中,质量矩阵通过递归算法计算:

\begin{equation}
\mathbf{M}(\boldsymbol{q}) = \sum_{i=1}^{n} \mathbf{J}_i^T(\boldsymbol{q}) \mathbf{I}_i \mathbf{J}_i(\boldsymbol{q})
\end{equation}

其中:
\begin{itemize}
    \item $\mathbf{J}_i(\boldsymbol{q})$:第 $i$ 个刚体的雅可比矩阵
    \item $\mathbf{I}_i$:第 $i$ 个刚体的空间惯性矩阵
\end{itemize}

\subsection{科氏力矩阵}

科氏力矩阵 $\mathbf{C}(\boldsymbol{q}, \dot{\boldsymbol{q}})$ 包含了科氏力和离心力效应。科氏力扭矩为:

\begin{equation}
\boldsymbol{\tau}_c = \mathbf{C}(\boldsymbol{q}, \dot{\boldsymbol{q}}) \dot{\boldsymbol{q}}
\end{equation}

在代码中,科氏力扭矩通过以下方式计算:

\begin{lstlisting}[language=C++, basicstyle=\ttfamily\small]
coriolis_torque_pin = coriolis_pin * dq_pin;
\end{lstlisting}

科氏力矩阵的计算基于质量矩阵的导数:

\begin{equation}
\mathbf{C}(\boldsymbol{q}, \dot{\boldsymbol{q}}) = \frac{1}{2} \left( \frac{\partial \mathbf{M}}{\partial \boldsymbol{q}} \dot{\boldsymbol{q}} + \mathbf{S}(\boldsymbol{q}, \dot{\boldsymbol{q}}) \right)
\end{equation}

其中 $\mathbf{S}(\boldsymbol{q}, \dot{\boldsymbol{q}})$ 是满足 $\dot{\mathbf{M}} = \mathbf{C} + \mathbf{C}^T$ 的矩阵。

\subsection{重力项}

重力项 $\boldsymbol{g}(\boldsymbol{q})$ 表示重力对各个关节产生的力矩。它通过以下公式计算:

\begin{equation}
\boldsymbol{g}(\boldsymbol{q}) = \sum_{i=1}^{n} \mathbf{J}_i^T(\boldsymbol{q}) m_i \boldsymbol{g}_0
\end{equation}

其中:
\begin{itemize}
    \item $m_i$:第 $i$ 个刚体的质量
    \item $\boldsymbol{g}_0$:重力加速度向量(在基座坐标系中)
\end{itemize}

在代码中,重力方向根据安装位置进行了变换:

\begin{lstlisting}[language=C++, basicstyle=\ttfamily\small]
// 安装位置:外旋 XYZ = 0 90 -90
Eigen::Matrix3d R_y = Eigen::AngleAxisd(M_PI/2, 
    Eigen::Vector3d::UnitY()).toRotationMatrix();
Eigen::Matrix3d R_z = Eigen::AngleAxisd(-M_PI/2, 
    Eigen::Vector3d::UnitZ()).toRotationMatrix();
Eigen::Matrix3d R_base_to_world = R_z * R_y;
Eigen::Vector3d gravity_world(0, 0, -9.81);
Eigen::Vector3d gravity_base = R_base_to_world.transpose() * gravity_world;
pinocchio_model.gravity.linear() = gravity_base;
\end{lstlisting}

\subsection{惯性力项}

惯性力项表示由于关节加速度产生的力矩:

\begin{equation}
\boldsymbol{\tau}_I = \mathbf{M}(\boldsymbol{q}) \ddot{\boldsymbol{q}}
\end{equation}

在代码中计算为:

\begin{lstlisting}[language=C++, basicstyle=\ttfamily\small]
inertia_torque_pin = mass_pin * ddq_pin;
\end{lstlisting}

\section{完整动力学项的计算}

在代码中,Pinocchio 计算的各项与动力学方程的关系如下:

\begin{align}
\boldsymbol{\tau}_{\text{full}} &= \boldsymbol{\tau}_I + \boldsymbol{\tau}_c + \boldsymbol{g} \\
&= \mathbf{M}(\boldsymbol{q}) \ddot{\boldsymbol{q}} + \mathbf{C}(\boldsymbol{q}, \dot{\boldsymbol{q}}) \dot{\boldsymbol{q}} + \boldsymbol{g}(\boldsymbol{q})
\end{align}

对应代码:

\begin{lstlisting}[language=C++, basicstyle=\ttfamily\small]
// 计算所有动力学项
pinocchio::computeAllTerms(pinocchio_model, pinocchio_data, q_pin, dq_pin);

// 获取各项
gravity_pin = pinocchio_data.g;  // 重力项
coriolis_pin = pinocchio_data.C;  // 科氏力矩阵
mass_pin = pinocchio_data.M;  // 质量矩阵
coriolis_torque_pin = coriolis_pin * dq_pin;  // 科氏力扭矩
inertia_torque_pin = mass_pin * ddq_pin;  // 惯性力扭矩
\end{lstlisting}

\section{负载处理}

在代码中,KWR75 力传感器的负载被添加到末端执行器:

\begin{lstlisting}[language=C++, basicstyle=\ttfamily\small]
// 创建负载惯性
Eigen::Vector3d load_com(load_centre[0], load_centre[1], load_centre[2]);
Eigen::Matrix3d inertia_matrix = Eigen::Matrix3d::Zero();
inertia_matrix(0, 0) = load_inertia[0];  // Ixx
inertia_matrix(1, 1) = load_inertia[1];  // Iyy
inertia_matrix(2, 2) = load_inertia[2];  // Izz
pinocchio::Inertia load_inertia_pin(load_mass, load_com, inertia_matrix);

// 将负载添加到末端执行器 joint 的惯性
pinocchio_model.inertias[ee_joint_id] += load_inertia_pin;
\end{lstlisting}

负载的惯性矩阵 $\mathbf{I}_{\text{load}}$ 为:

\begin{equation}
\mathbf{I}_{\text{load}} = \begin{bmatrix}
I_{xx} & 0 & 0 \\
0 & I_{yy} & 0 \\
0 & 0 & I_{zz}
\end{bmatrix}
\end{equation}

其中 $I_{xx}$、$I_{yy}$、$I_{zz}$ 是主惯量(假设负载的主轴与坐标系对齐)。

\section{计算效率}

Pinocchio 使用高效的递归算法(基于 Featherstone 的算法)计算动力学项,计算复杂度为 $O(n)$,其中 $n$ 是机器人的自由度。这使得它非常适合实时控制应用。

\section{与 xMateModel 的对比}

在混合力位控制代码中,Pinocchio 的计算结果与 xMateModel 的计算结果进行对比:

\begin{align}
\Delta \boldsymbol{g} &= \boldsymbol{g}_{\text{xMate}} - \boldsymbol{g}_{\text{Pinocchio}} \\
\Delta \boldsymbol{\tau}_c &= \boldsymbol{\tau}_{c,\text{xMate}} - \boldsymbol{\tau}_{c,\text{Pinocchio}} \\
\Delta \boldsymbol{\tau}_I &= \boldsymbol{\tau}_{I,\text{xMate}} - \boldsymbol{\tau}_{I,\text{Pinocchio}} \\
\Delta \boldsymbol{\tau}_{\text{full}} &= \boldsymbol{\tau}_{\text{full,xMate}} - \boldsymbol{\tau}_{\text{full,Pinocchio}}
\end{align}

这些差值可以帮助验证:
\begin{itemize}
    \item 模型参数的一致性(质量、惯性、质心位置)
    \item 重力方向的设置是否正确
    \item 负载参数是否正确添加
    \item 数值计算的精度
\end{itemize}

\section{总结}

Pinocchio 提供了高效的动力学计算功能,通过 \texttt{computeAllTerms()} 函数可以一次性计算:
\begin{itemize}
    \item 质量矩阵 $\mathbf{M}(\boldsymbol{q})$
    \item 科氏力矩阵 $\mathbf{C}(\boldsymbol{q}, \dot{\boldsymbol{q}})$
    \item 重力项 $\boldsymbol{g}(\boldsymbol{q})$
\end{itemize}

这些项可以进一步组合得到:
\begin{itemize}
    \item 惯性力扭矩:$\boldsymbol{\tau}_I = \mathbf{M} \ddot{\boldsymbol{q}}$
    \item 科氏力扭矩:$\boldsymbol{\tau}_c = \mathbf{C} \dot{\boldsymbol{q}}$
    \item 总动力学扭矩:$\boldsymbol{\tau}_{\text{full}} = \boldsymbol{\tau}_I + \boldsymbol{\tau}_c + \boldsymbol{g}$
\end{itemize}

在混合力位控制中,这些计算结果用于:
\begin{enumerate}
    \item 验证 xMateModel 的计算结果
    \item 调试动力学模型参数
    \item 分析控制性能
\end{enumerate}

\end{document}


